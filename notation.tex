\usepackage{amssymb,bbm}
\usepackage{unicode-math}

\defineshorthand{"~}{\babelhyphen{nobreak}}
\AtBeginDocument{\useshorthands*{"}}

\tikzset{>=latex,radius=\pgfplotmarksize}
\pgfkeys{/pgf/number format/set thousands separator={\,}}
\pgfplotsset{compat=1.16,cycle list name=black white,colormap name=viridis}


\newlist{codelisting}{enumerate}{4}
\setlist[codelisting,1]{label=\arabic*:,ref=\arabic*}
\setlist[codelisting,2]{label=\thecodelistingi.\arabic*:,ref=\thecodelistingi.\arabic*}
\setlist[codelisting,3]{label=\thecodelistingii.\arabic*:,ref=\thecodelistingii.\arabic*}
\setlist[codelisting,4]{label=\thecodelistingiii.\arabic*:,ref=\thecodelistingiii.\arabic*}


\DeclareMathOperator{\asNat}{asInt}
\DeclareMathOperator{\asReal}{asReal}
\DeclareMathOperator{\asStr}{asStr}
\DeclareMathOperator{\bigO}{\symcal{O}}
\DeclareMathOperator{\dom}{dom}
\DeclareMathOperator*{\Ex}{E}
\DeclareMathOperator{\gap}{gap}
\DeclareMathOperator{\ic}{ic}
\DeclareMathOperator{\KC}{K}
\DeclareMathOperator{\littleo}{\symcal{o}}
\DeclareMathOperator{\pc}{pc}
\let\Pr\relax	% Just P for probability
\DeclareMathOperator*{\Pr}{P}
\DeclareMathOperator{\prefix}{prefix}
\DeclareMathOperator{\rank}{rank}
\DeclareMathOperator{\soph}{soph}

\AtBeginDocument{
  \renewcommand{\epsilon}{\varepsilon}
  \renewcommand{\mu}{{\symup{\mupmu}}}	% Do not use \mu as a variable
  \renewcommand{\setminus}{\smallsetminus}
}
\newcommand{\binary}{\mathbbm{2}}
\newcommand{\bbN}{\mathbbm{N}}
\newcommand{\calF}{{\symcal{F}\kern-.2em}}
\newcommand{\calL}{\symcal{L}}
\DeclareMathOperator{\updelta}{\symup{\mupdelta}}
\DeclareMathOperator{\upM}{\symup{M}}
\DeclareMathOperator{\upN}{\symup{N}}

\newcommand{\bits}[1]{\textnormal{\texttt{#1}}}
\newcommand{\cl}[1]{\textnormal{\textbf{#1}}}	% Complexity classes
\newcommand{\cladv}[2]{\ensuremath{\textnormal{\textbf{#1}}\!_\textnormal{\textbf{/#2}}}}	% Complexity classes with advice
\newcommand{\clnu}[1]{\cl{#1$_\textbf{nu}$}}	% Nonuniform complexity classes
\newcommand{\cltime}[1]{\cl{TIME(#1)}}	% Deterministic time classes
\newcommand{\clX}[1]{\cl{\textit{X}#1}}	% Slicewise complexity classes
\newcommand{\clXnu}[1]{\clnu{\textit{X}#1}}	% Nonuniform slicewise complexity classes
\newcommand{\code}{\textsf}
\newcommand{\Cpp}{C\nolinebreak\hspace{-.05em}\raisebox{.4ex}{\tiny\bf +}\nolinebreak\hspace{-.10em}\raisebox{.4ex}{\tiny\bf +}}
\newcommand{\deq}{=}	% Defined equal
\newcommand{\given}{\mid}
\newcommand{\itemcont}{\\\hspace*{1em}}
\newcommand{\length}[1]{{\lvert#1\rvert}}
\newcommand{\num}{\pgfmathprintnumber[fixed]}
\newcommand{\pair}[2]{{\langle #1, #2\rangle}}
\newcommand{\pdash}{\textsl{p}\babelhyphen{nobreak}}	% Polynomial time functions
\newcommand{\pr}[1]{\textnormal{\textsc{#1}}}	% Problems
\newcommand{\quasile}{\preccurlyeq}
\newcommand{\notquasile}{\npreccurlyeq}
\newcommand{\quasilenu}{\quasile_\mathrm{nu}}
\newcommand{\reland}{\;\text{ and }\;}	% Relating conditions conjunctively
\newcommand{\st}[1][]{\;#1|\;}	% Such that (with optional size)
\newcommand{\symdiff}{\bigtriangleup}	% Symmetric difference

\newcommand{\immune}[1]{almost #1\babelhyphen{nobreak}bi-immune}
\newcommand{\levelable}[1]{#1\babelhyphen{nobreak}omni-levelable}

% vim:ts=50
