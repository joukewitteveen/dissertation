\section{as Polynomial Levelability}

The computational complexity of a set is focussed on the complexity of its hardest instances, or on that of the majority of its instances.
This has been the case since the identification of efficient computability by Cobham and Edmonds around 1965 \parencite[see][]{goldreich2008computational}.
Intractability results \parencite{cook1971complexity,garey1979computers} generally consider the hardest instances, while average case complexity looks at complexity of the bulk.
Yet, even very hard sets can have simple instances and often lots of them.

The indiscriminate judgement of the computational complexity of sets was addressed by \textcite{lynch1975reducibility} in \citeyear{lynch1975reducibility}.
She dealt with the distribution of complexity inside sets by examining intrinsically hard subsets.
Although fixed-parameter tractability would only be established some two decades later \parencite{downey1992fixed}, a parameterized computational complexity theory was thus started.
In this section, we shall built a parameterized theory of computational complexity on these early foundations.
While the traditional parameterized classes are easily recovered, our analysis will be concerned with parameterizations as independent measures of complexity.
Much of our theory revolves around collections of parameterizations that put a given set in a parameterized complexity class.
Such collections function as an interface to the complexity of the sets that gave rise to them.
Moreover, they reveal that no practically fixed-parameter tractable sets have optimal parameterizations.

\subsection{Stratified Computational Complexity}
The intrinsically hard parts of a set examined by \citeauthor{lynch1975reducibility} are those on which no decision procedure can run in polynomial time infinitely often.
\begin{definition}
  A decidable set $C$ is a \defkeyat{core!polynomial}{polynomial core} of a set $A$ if for every polynomial approximation $\phi$ for $A$ the intersection $C \cap \dom(\phi)$ is finite.
\end{definition}

We shall give a name to the dual as well.
\begin{definition}
  A set $C$ is a \defkeyat{segment@$t$-segment!polynomial}{polynomial segment} of a set $A$ if there is a polynomial approximation $\phi$ for $A$ such that $C = \dom(\phi)$.

  Replacing the polynomial designator by a function $t$ or class of functions $\bigO(f)$, we obtain the definitions of a \defkeyat{segment@$t$-segment}{$t$-segment} and an \defkeyat{segment@$t$-segment!O(f)@$\bigO(f)$}{$\bigO(f)$-segment}.
\end{definition}
Thus, a set is a polynomial core if it has a finite intersection with every polynomial segment.

Subsets of polynomial cores are polynomial cores for the same set, as are finite variations.
This complicates thinking of the members of a core as inherently hard instances of a set.
Any specific instance can arbitrarily be made part of, or excluded from a polynomial core.
However, for some sets the collection of polynomial cores contains a maximal element with respect to inclusion up to finite variation.
If this is the case, the set is split into an easy part and a hard part.
\begin{theorem}
\label{thm:maximal}
  A set has a maximal (up to finite variations) polynomial core if and only if it has a maximal polynomial segment.
\end{theorem}
\begin{proof}
  $\Longleftarrow$.
  The complement of a maximal polynomial segment of a set is a polynomial core of that set that cannot be infinitely extended.
  Hence, it is a maximal polynomial core.

  $\Longrightarrow$.
  We claim that the complement of a maximal polynomial core of a set $A$ is a polynomial segment of $A$.
  Let $S_1, S_2, S_3, \ldots$ be an enumeration of the, at most countably many, polynomial segments of $A$.
  Assume now that the complement of some maximal polynomial core of $A$ is not a polynomial segment of $A$.
  Then, for all $j$, there are infinitely many elements in this complement outside the polynomial segment $\bigcup_{i \le j} S_i$.
  Consequently, we would be able to extend the maximal polynomial core with infinitely many elements, one for each $j$, contradicting its maximality.
  Hence, the complement of a maximal polynomial core of $A$ must be a polynomial segment of $A$.
\end{proof}

It follows that the complement of a maximal polynomial segment is a maximal polynomial core.
Because polynomial segments are necessarily in \cl{P}, we get the following.
\begin{corollary}
  A maximal polynomial core, if it exists, is in \cl{P}.
\end{corollary}

Of course, within a maximal polynomial core $C$ of a set $A$, membership of $A$ is hard to decide.
While $C$ is in \cl{P}, no polynomial approximation for $A$ decides membership of $A$ for infinitely many elements of $C$.

A set is outside \cl{P} precisely when it has an infinite polynomial core \parencite{lynch1975reducibility}.
Some sets are so far removed from \cl{P} that they have no infinite polynomial segments.
This is a form of immunity against \cl{P}.
By Theorem~\ref{thm:maximal}, we obtain an elegant definition in terms of polynomial cores.
This definition is nonstandard, but its equivalence to the common definition based on infinite subsets was already observed by \textcite{balcazar1985bi-immune} \parencite[see also][]{book1988polynomial}.
\begin{definition}
  A set $A$ is \defkeyat{bi-immune!for P@for \cl{P}}{\cl{P}-bi-immune} if $\binary^+$ is a polynomial core for $A$.
\end{definition}

Thus a set is \cl{P}-bi-immune if it has the largest polynomial core possible.
This definition is easily generalized to sets that have a maximal polynomial core.
In this case the equivalence of our definition to the more common definition of the same term as a disjoint union was observed by \textcite{orponen1986classification}.
\begin{definition}
  A set is \defkeyat{almost bi-immune!for P@for \cl{P}}{\immune{\cl{P}}} if it has a maximal polynomial core.
\end{definition}
Note that, by our definitions, sets in \cl{P}, and finite sets in particular, have maximal polynomial cores and are therefore \immune{\cl{P}}.
This may seem somewhat peculiar, but is in agreement with the definitions used by \textcite{orponen1986classification} and of no objection in the sequel.

With polynomial segments and polynomial cores we make no distinction between the members and the nonmembers of a set.
Segments as well as cores of a set are maximal if they cannot be extended by infinitely many members or nonmembers of the set.
In \parencite{orponen1985polynomial,orponen1986optimal}, sets of which every polynomial segment can be extended by infinitely many \emph{members} into a larger polynomial segment are called \emph{\cl{P}-levelable}.
For our analysis, we need a more general definition that covers sets of which every polynomial segment can be extended by infinitely many arbitrary instances.
While tempting, such sets should not be called \cl{P}-bi-levelable.
That name would more naturally describe sets of which every polynomial segment can be extended by both infinitely many members \emph{and} infinitely many nonmembers.
\begin{definition}
  A set is \defkeyat{omni-levelable!for P@for \cl{P}}{\levelable{\cl{P}}} if it has no maximal polynomial segment.
\end{definition}
By Theorem~\ref{thm:maximal} and in line with the work of \textcite{orponen1985polynomial}, a set is \levelable{\cl{P}} precisely when it is not \immune{\cl{P}}.
Furthermore, every \cl{P}-levelable set is \levelable{\cl{P}}.
It was observed by \textcite{orponen1986optimal} that very many (natural) intractable sets are \cl{P}-levelable.
As a consequence, we find that there are many \levelable{\cl{P}} sets.

Especially when a set is \levelable{\cl{P}}, it makes sense to look at the collection of its polynomial segments.
In order to uncover connections between segments, we may turn to collections of sets and ask whether they are collections of polynomial segments.
Thus sets are related to collections of their polynomial segments.
We can think of this relation as the definition of a complexity class that lifts \cl{P} to the realm of parameterized analysis.
One way to formalize this lifting is by means of the nonuniform operator $\clXnu{}$.
\begin{definition}
  A set $A$ is in \defkeyat{XPnu@\clXnu{P}}{\clXnu{P}} with parameterization $\eta$ if for every parameter value $k$ the set $\eta_k$ is a polynomial segment of $A$.
\end{definition}

Lifting \cl{P} to different fields of analysis is not new and several ways of doing so have been studied previously.
Using $\exists$ as an operator, the nondeterministic counterpart, \cl{NP}, of \cl{P} has been obtained as \cl{$\exists$P}.
In probabilistic complexity theory, the \cl{\textit{BP}}~operator was derived from the complexity class \cl{\textit{BP}P} by \textcite{schoning1989probabilistic}.
This operator was then used to define various probabilistic complexity classes based on their deterministic counterparts.

The class \clXnu{P} is nonuniform in two ways.
Firstly, a set may be in \clXnu{P} without there being a procedure to instantiate polynomial approximations from their corresponding parameter values.
Secondly, even if there was, the parameter dependence of the polynomial time bounds of the approximations may not be of a computational nature.
These considerations spur a fully uniform alternative to the nonuniform \clXnu{} operator.
Corresponding parameterized complexity classes are called \emph{strongly uniform} by \textcite{downey1999parameterized}.
\begin{definition}
\label{def:xp}
  A set $A$ is in \defkeyat{XP@\clX{P}}{\clX{P}} with parameterization $\eta$ if there is a direct parameterized procedure $\phi$ and a computable function $f$ satisfying
  \begin{itemize}
  \item $\eta$ is the parameterization corresponding to $\phi$, and
  \item for every parameter value $k$, with $t_k$ mapping $n$ to $f(k) \cdot n^{f(k)}$, the partial application of $\phi$ to $k$ yields a $t_k$-approximation for $A$.
  \end{itemize}
\end{definition}

Note that we indicate nonuniformity with the \clnu{}-subscript, but use no modifier for the fully uniform case.
In other places, e.g.~\url{https://complexityzoo.uwaterloo.ca/Complexity_Zoo:X}, special notation is used instead to denote uniform classes.

In the current century, the class \clX{P} has been called \emph{slicewise~\cl{P}}, giving a name to the \clX{}~operator \parencite{flum2003describing}.
However, in earlier work, \textcite{downey1999parameterized} used ``slicewise \cl{P}'' to denote a different complexity class, namely \cl{FPT}.
The definition of \clX{P} is very permissive in that the exponent in the running time of the polynomial approximations may be unbounded.
In \cl{FPT}, this freedom is restricted.
\begin{definition}
  A set $A$ is in \defkeyat{FPTnu@\clnu{FPT}}{\clnu{FPT}} with parameterization $\eta$ if there is a polynomial $p$ such that for every parameter value $k$ the set $\eta_k$ is an $\bigO(p)$-segment of $A$.
\end{definition}

Sets in \cl{FPT} are said to be \emph{fixed-parameter tractable}.
The slices of a parameterization with which a set is in \clnu{FPT} are polynomial segments where the degree of the polynomial is bounded.
Observe that this amounts to a reversal of quantifiers when compared to \clnu{XP}.
With \clnu{FPT} there is a single polynomial that is used for all parameter values.
On the other hand, with \clXnu{P} every parameter value may have a different polynomial associated to it.
Since we have $\cl{P} = \bigcup_c \cltime{$n^c$}$, we can express this reversal of quantifiers symbolically as
\label{eq:xpfpt}
\begin{align*}
  \clXnu{P}	&= \clXnu{$\Big(\bigcup_c \cltime{$n^c$}\Big)$}
\shortintertext{and}
  \clnu{FPT}	&= \bigcup_c \Big(\clXnu{\cltime{$n^c$}}\Big).
\end{align*}

The fully uniform version of these equalities directs us to the definition of fixed-parameter tractability.
\begin{definition}
\label{def:fpt}
  A set $A$ is in \defkeyat{FPT@\cl{FPT}}{\cl{FPT}} with parameterization $\eta$ if there is a direct parameterized procedure $\phi$, a computable function $f$, and a polynomial $p$ satisfying
  \begin{itemize}
  \item $\eta$ is the parameterization corresponding to $\phi$, and
  \item for every parameter value $k$, the partial application of $\phi$ to $k$ yields a $(f(k) \cdot p)$-approximation for $A$.
  \end{itemize}
\end{definition}

When a set $A$ is in a parameterized complexity class, for example \cl{FPT}, with a parameterization $\eta$, we write $(A, \eta) \in \cl{FPT}$.
We remark that members of (nonuniform) \clX{P} need not be in (nonuniform) \cl{FPT} \parencite{downey1999parameterized,flum2006parameterized}.
Likewise, the classes \clX{P} and \cl{FPT} are strictly smaller than their nonuniform counterparts, \clXnu{P} and \clnu{FPT} \parencite{downey1993fixed}.
Many sets have been found to be fixed-parameter tractable \parencite{downey1995fixed,niedermeier2006invitation,cygan2015parameterized}.
Moreover, parameter values associated with typical instances are small in practice \parencite{downey1999framework,downey1999parameterized}.
In that way, fixed-parameter tractability is a successful notion of efficient computability.

\subsection{Order Theory for Parameterizations}
We are now in a position to study the distribution of complexity inside a set in terms of the parameterizations that put the set in one of our classes.
To this end, two collections of parameterizations that put a given set in some parameterized complexity class are of central importance.
\begin{definition}
  Given a set $A$ and a parameterized complexity class \cl{C}, we denote the collection of parameterizations with which $A$ is in \cl{C}\index{F@$\calF_{(A, \cl{C})}$} by
  \begin{align*}
    \calF_{(A, \cl{C})} &\deq \{\eta \st (A, \eta) \in \cl{C}\}.
    \intertext{More generally, we consider the parameterizations in light of a parameterized complexity class irrespective of a particular set\index{L@$\calL_\cl{C}$},}
    \calL_\cl{C} &\deq \{\eta \st \exists A\colon (A, \eta) \in \cl{C}\}.
  \end{align*}
\end{definition}

As with polynomial cores, we are not so much interested in finite variations on the slices of a parameterization.
Instead, we are mostly interested in parameterizations of which the slices grow in infinitely large steps.
\begin{definition}
  A parameterization $\eta$ has \defkey{imix} (infinitely many infinite extensions) if for every parameter value $k$ there is a parameter value $k'$ such that the set $\eta_{k'} \backslash \eta_k$ is infinite.
\end{definition}

Recall that a parameterization is a directed cover.
Because of that, the existence of a slice that has infinitely many members not in $\eta_k$ implies the existence of an infinite superset of $\eta_k$.

Parameterizations can be compared using the minimization function.
Thinking of a parameterization $\eta$ as embodying a rate of convergence, a lower value of $\mu_\eta$ is indicative of a faster convergence.
Informally, we may compare two parameterizations by looking at bounds on the convergence rate of one in terms of the convergence rate of the other.
Thus, consider the required minimum length of a parameter value in one parameterization for instances of a bounded parameter length in another parameterization.
\begin{definition}
  Given parameterizations $\eta$ and $\zeta$, the \defkeyat{gap@$\gap_{\eta, \zeta}$}{gap} function $\bbN \to \bbN \cup \{\infty\}$, is defined as
  \begin{equation*}
    \gap_{\eta, \zeta}(n) \deq \max\{\mu_{\eta}(x) \st x \in \binary^+ \reland \mu_{\zeta}(x) \le n\},
  \end{equation*}
  where we take the maximum of the empty set to be $0$.
\end{definition}

Comparing parameterizations using this gap function enables us to define a nonuniform and a uniform order on parameterizations.
A parameterization $\eta$ is below another, when a bound on the point of convergence for the other parameterization can be turned into such a bound for $\eta$.
Similar orders have been considered by \textcite{komusiewicz2012new} and \textcite{fellows2013towards}.

\begin{definition}
  A parameterization $\eta$ is below a parameterization $\zeta$ in the \emph{nonuniform} order \defkeyat{<nu@$\quasilenu$}{$\quasilenu$} if for all $n$ we have $\gap_{\eta, \zeta}(n) < \infty$.
\end{definition}

From a computational standpoint, a bound on the gap between two parameterizations is only useful if it is computable.
By Theorem~\ref{thm:decidable}, the uniform variant of the order on parameterizations is of interest for parameterizations arising from direct parameterized procedures.

\begin{definition}
  A parameterization $\eta$ is below a parameterization $\zeta$ in the \emph{uniform} order \defkeyat{<@$\quasile$}{$\quasile$} if there is a computable function upper bounding $\gap_{\eta, \zeta}$.
\end{definition}

We observe the relationship $\quasile \subset \quasilenu$ between these orders.
The orders provide structure to the class of parameterizations.
\begin{lemma}
\label{lem:preorder}
  Both $\quasilenu$ and $\quasile$ are reflexive and transitive orders on the class of parameterizations.
\end{lemma}
\begin{proof}
  Reflexivity follows from the observation that for every parameterization $\eta$ the gap function $\gap_{\eta, \eta}$ is bounded by the identity function.
  For transitivity, two remarks suffice.
  Firstly, the composition of finite bounding functions is again a finite bounding function.
  Secondly, the composition of computable functions is computable.
\end{proof}

Neither order is antisymmetric.
Therefore, it is convenient to work with the associated partially ordered set of \emph{equivalence classes} instead of with parameterizations directly.
Note that every parameterization with imix is unequal to any parameterization without imix, both using the nonuniform as well as using the uniform order.

\begin{lemma}
\label{lem:imix}
  Let $\eta$ and $\zeta$ be parameterizations, where $\eta$ has imix but $\zeta$ does not.
  One of $\eta \quasilenu \zeta$ and $\zeta \quasilenu \eta$ fails to hold (and similarly for $\quasile$).
\end{lemma}
\begin{proof}
  The statement for the uniform order follows from that for the nonuniform order by the inclusion $\quasile \subset \quasilenu$.

  Let $m$ be so that no parameter value of length at least $m$ is associated with a slice of $\zeta$ that has an infinite extension.
  Such an $m$ can be found for every parameterization that does not have imix, because parameterizations are directed.

  Suppose we have $\eta \quasilenu \zeta$.
  There is then a parameter value $k$ such that $\eta_k$ is a superset of all slices $\zeta_j$ for which we have $\length{j} \le m$.
  Since $\eta$ has imix, there exists a parameter value $k'$ such that $\eta_{k'} \backslash \eta_k$ is infinite.
  We find that $\zeta \quasilenu \eta$ fails because $\gap_{\zeta, \eta}(\length{k'})$ must be infinite.

  Suppose we have $\zeta \quasilenu \eta$.
  If $\eta \quasilenu \zeta$ were to hold as well, $\gap_{\eta, \zeta}(m)$ would be finite.
  However, there would then be a parameter value $k'$ such that $\eta_{k'}$ is infinitely larger than any $\zeta_j$, when we have $\length{j} \le m$.
  As $\zeta$ does not have imix, this would violate the assumed $\zeta \quasilenu \eta$.
\end{proof}

In the context of orders on parameterizations, we refer to the equivalence classes of parameterizations when speaking simply of parameterizations.
As we have seen in Lemma~\ref{lem:imix}, parameterizations with imix will remain distinct from those without when employing this convention.

\subsubsection{Nonuniform Complexity Classes}
When we look at $\calL_\clXnu{P}$ or $\calL_\clnu{FPT}$ in terms of equivalence classes ordered by $\quasilenu$, we recognize a familiar \parencite{davey2002introduction} structure.
\begin{theorem}
\label{thm:nulattice}
  Ordered by $\quasilenu$, the equivalence classes in $\calL_\clXnu{P}$ and $\calL_\clnu{FPT}$ form bounded distributive lattices.
\end{theorem}
\begin{proof}
  Observe that the slices of a parameterization in $\calL_\clXnu{P}$ are in \cl{P}.
  Likewise, for the slices of a parameterization in $\calL_\clnu{FPT}$ there is a constant $c$ such that the slices are in \cltime{$n^c$}.
  The converses of both statements are witnessed by the empty set.
  The empty set is put in \clXnu{P} and \clnu{FPT} respectively by parameterizations that meet said descriptions.

  For the current proof we shall focus on $\calL_\clnu{FPT}$.
  The proof for $\calL_\clXnu{P}$ is slightly less involved.

  \paragraph{$\calL_\clnu{FPT}$ is bounded.}
  A least element of $\calL_\clnu{FPT}$ is present in the form of the parameterization consisting of `full' slices only, $(\binary^+)_{k \in \binary^+}$.
  Indeed, any constant bounds the gap from this parameterization to another and thus this parameterization is below any other with respect to $\quasilenu$.

  A greatest element can be found in any parameterization of which the slices are all finite.
  Take, for instance, the length parameterization, $(\{x \st \length{x} \le \asNat(k)\})_{k \in \binary^+}$.
  Regardless of the bound on the length of parameter values, the gap from any parameterization to this parameterization is the maximum of a finite set.
  The gap is therefore always finite.
  Hence, this parameterization is above any other with respect to $\quasilenu$.

  \paragraph{$\calL_\clnu{FPT}$ contains greatest lower bounds.}
  Given two parameterizations, $\eta \in \calL_\clnu{FPT}$ and $\eta' \in \calL_\clnu{FPT}$, we can construct a new parameterization that acts as a greatest lower bound for the two.
  Define this new parameterization as
  \begin{equation*}
    \zeta \deq (\{x \st x \in \eta_k \lor x \in \eta'_{k'}\})_{\pair{k}{k'} \in \binary^+}.
  \end{equation*}
  So, every slice of $\zeta$ is the union of a slice of $\eta$ and a slice of $\eta'$, and we have $\zeta_\pair{k}{k'} = \eta_k \cup \eta'_{k'}$.
  This construction ensures that $\zeta$ is a point-cofinite directed cover and with that, a parameterization.
  Moreover, if, for some $c$, all slices of $\eta$ and all slices of $\eta'$ are in \cltime{$n^c$}, then all slices of $\zeta$ are in \cltime{$n^c$}.
  As $\eta$ and $\eta'$ were picked from $\calL_\clnu{FPT}$, we find that $\zeta$ must also be a member of $\calL_\clnu{FPT}$.

  By construction, every slice of either $\eta$ or $\eta'$ is included in a slice of $\zeta$.
  It follows that the gap from $\zeta$ to both $\eta$ and $\eta'$ is finite and thus that $\zeta$ is below each of them.
  It remains to show that $\zeta$ is the greatest among such lower bounds.
  For this, suppose that some $\zeta'$ too is below both $\eta$ and $\eta'$ and consider $\gap_{\zeta'\!, \zeta}$.
  Because our pairing function is length increasing in both arguments, we have, for every $n$,
  \begin{equation*}
    \gap_{\zeta'\!, \zeta}(n) \le \max\{\gap_{\zeta'\!, \eta}(n),\ \gap_{\zeta'\!, \eta'}(n)\}.
  \end{equation*}
  As both elements of the set on the right hand side are finite by assumption, we find that the gap from $\zeta'$ to $\zeta$ is finite for all values of $n$.
  Thus, $\zeta'$ is below $\zeta$, and $\zeta$ is a greatest lower bound for $\eta$ and $\eta'$ with respect to $\quasilenu$, as desired.

  \paragraph{$\calL_\clnu{FPT}$ contains least upper bounds.}
  In a similar vein, the existence of least upper bounds can be shown.
  Taking the slicewise intersection of two given parameterizations, $\eta$ and $\eta'$, we obtain
  \begin{equation*}
    \zeta \deq (\{x \st x \in \eta_k \land x \in \eta'_{k'}\})_{\pair{k}{k'} \in \binary^+}.
  \end{equation*}
  Like the parameterization constructed previously, this defines a point-cofinite directed cover, thus, indeed, a parameterization.
  When $\eta$ and $\eta'$ are members of $\calL_\clnu{FPT}$, this parameterization is a member of $\calL_\clnu{FPT}$ too.

  Using that our pairing function is length increasing, we find, for all $n$, that $\gap_{\eta, \zeta}(n)$ and $\gap_{\eta'\!, \zeta}(n)$ are both at most $n$.
  Thus, $\zeta$ is an upper bound to $\eta$ and $\eta'$.
  To see that it is a least upper bound, suppose that $\zeta'$ be another upper bound to $\eta$ and $\eta'$.
  Our pairing function is so that, for all $k$ and $k'$, we have $\length{\pair{k}{k'}} \le 2(\length{k} + \length{k'})$.
  Because of this, we also have, for all $n$,
  \begin{equation*}
    \gap_{\zeta, \zeta'}(n) \le 2(\gap_{\eta, \zeta'}(n) + \gap_{\eta'\!, \zeta'}(n)).
  \end{equation*}
  Since the right hand side of this equation is finite, we can conclude that we have $\zeta \quasilenu \zeta'$.
  Accordingly, $\zeta$ is a least upper bound for $\eta$ and $\eta'$ with respect to $\quasilenu$.

  \paragraph{$\calL_\clnu{FPT}$ is distributive.}
  Let $\eta$ and $\eta'$ be two parameterizations in $\calL_\clnu{FPT}$.
  As is customary, we denote their greatest lower bound by $\eta \land \eta'$ and their least upper bound by $\eta \lor \eta'$.
  Beware that in the above construction, $\eta \land \eta'$ was based on a disjunction $x \in \eta_k \lor x \in \eta'_{k'}$ and vice versa for $\eta \lor \eta'$.
  Also, recall that in this context, we use parameterizations in place of their equivalence classes according to $\quasilenu$.
  We shall show that, for any three parameterizations $\eta$, $\eta'$, and $\eta''$ in $\calL_\clnu{FPT}$, the equivalence class of
  \begin{align*}
    \zeta &\deq \eta \lor (\eta' \land \eta'')
    \shortintertext{equals that of}
    \zeta' &\deq (\eta \lor \eta') \land (\eta \lor \eta'').
  \end{align*}
  The dual, where $\land$ and $\lor$ are interchanged, is implied \parencite{davey2002introduction}.

  By our constructions of greatest lower bounds and least upper bounds, we may interpret a parameter value for $\zeta$ as a triplet $\pair{i}{\pair{j}{k}}$.
  If an instance $x$ is in a slice $\pair{i}{\pair{j}{k}}$ of $\zeta$, it is also in slice $\pair{\pair{i}{j}}{\pair{i}{k}}$ of $\zeta'$.
  Made explicit, this comes down to the logical implication
  \begin{equation*}
    x \in \eta_i \land (x \in \eta'_j \lor x \in \eta''_k) \implies (x \in \eta_i \land x \in \eta'_j) \lor (x \in \eta_i \land x \in \eta''_k).
  \end{equation*}
  Our pairing function is so that $\length{\pair{i}{\pair{j}{k}}}$ is at least $\length{i} + \length{j} + \length{k}$ and $\length{\pair{\pair{i}{j}}{\pair{i}{k}}}$ is at most $4(2\length{i} + \length{j} + \length{k}) \le 8(\length{i} + \length{j} + \length{k})$.
  It follows that, for all $n$, we have $\gap_{\zeta'\!, \zeta}(n) \le 8n$.
  Hence, it follows that we have $\zeta' \quasilenu \zeta$.

  For the converse, observe that if an instance $x$ is in a slice $\pair{\pair{i_1}{j}}{\pair{i_2}{k}}$ of $\zeta'$, it must also be in slice $\pair{i_1}{\pair{j}{k}}$ or in slice $\pair{i_2}{\pair{j}{k}}$ of $\zeta$.
  This follows from the logical implication
  \begin{equation*}
    (x \in \eta_{i_1} \land x \in \eta'_j) \lor (x \in \eta_{i_2} \land x \in \eta''_k) \implies \begin{cases}
      x \in \eta_{i_1} \land (x \in \eta'_j \lor x \in \eta''_k)\text{ or} \\
      x \in \eta_{i_2} \land (x \in \eta'_j \lor x \in \eta''_k).
    \end{cases}
  \end{equation*}
  An argument concerning length bounds of our pairing function similar to the one used before now gets us a bound on the gap from $\zeta$ to $\zeta'$.
  For all $n$, we have $\gap_{\zeta, \zeta'}(n) \le 4n$.
  Thus we also have $\zeta \quasilenu \zeta'$ and both parameterizations are in the same equivalence class.
\end{proof}

For this proof, it is important that the definition of a parameterization makes no strong demands on the inclusion order of slices.
Most expositions of parameterized computational complexity theory appear to tacitly assume that the inclusion order is a linear order.
In the most prevalent parameterizations, parameter values are interpreted as numbers.
The inclusion order on the slices of these parameterizations follows the standard numerical order on the corresponding parameter values.
However, in the previous proof, a linear inclusion order on slices is not preserved by the constructions of greatest lower bounds and least upper bounds.

The appeal of our nonuniform order on parameterizations is provided by a companion of Theorem~\ref{thm:nulattice}.
For every set $A$, the collections $\calF_{(A, \clXnu{P})}$ and $\calF_{(A, \clnu{FPT})}$ sit nicely \parencite{davey2002introduction} inside $\calL_\clXnu{P}$ and $\calL_\clnu{FPT}$, respectively.
\begin{theorem}
\label{thm:nufilter}
  Let $A$ be a set.
  Ordered by $\quasilenu$, the collections $\calF_{(A, \clXnu{P})}$ and $\calF_{(A, \clnu{FPT})}$ are filters in $\calL_\clXnu{P}$ and $\calL_\clnu{FPT}$, respectively.
\end{theorem}
\begin{proof}
  Like with Theorem~\ref{thm:nulattice}, we present a proof for the \clnu{FPT} case only.
  All ingredients of a proof for the \clXnu{P} case are included in this proof.

  \paragraph{$\calF_{(A, \clnu{FPT})}$ in nonempty.}
  For every polynomial $p$, every finite set is an $\bigO(p)$-segment of $A$.
  Therefore, every parameterization of which all slices are finite is in $\calF_{(A, \clnu{FPT})}$.
  An example of such a parameterization is the length parameterization.

  \paragraph{$\calF_{(A, \clnu{FPT})}$ is upward closed.}
  Let $\eta$ be a member of $\calF_{(A, \clnu{FPT})}$ and $\zeta$ be any parameterization in $\calL_\clnu{FPT}$ for which we have $\eta \quasilenu \zeta$.
  Furthermore, let $c$ be so that the slices of $\zeta$ are all in \cltime{$n^c$}.
  It suffices to prove the existence of a polynomial $p$ such that the slices of $\zeta$ are $\bigO(p)$-segments of $A$.

  Since we have $\eta \quasilenu \zeta$, given a parameter value $j$, there is a parameter value $k$ such that we have $\zeta_j \subseteq \eta_k$.
  Also, as $\eta$ is in $\calF_{(A, \clnu{FPT})}$, there is a polynomial $q$ such that any slice of $\eta$ is the domain of an $\bigO(q)$-approximation for $A$.
  An approximation for $A$ that runs the approximation with domain $\eta_k$, but only for members of $\zeta_j$ can be crafted.
  With $p$ mapping $n$ to $n^c + q(n)$, the running time of such an approximation can be kept in $\bigO(p)$.
  This $p$ is a polynomial that is independent of the parameter value $j$, so it meets our requirements.

  \paragraph{$\calF_{(A, \clnu{FPT})}$ contains greatest lower bounds.}
  Given parameterizations $\eta$ and $\eta'$ taken from $\calF_{(A, \clnu{FPT})}$, consider the greatest lower bound $\zeta$ as constructed in the proof of Theorem~\ref{thm:nulattice}.
  There are polynomials $q$ and $q'$ such that every slice of $\eta$ is an $\bigO(q)$-approximation for $A$ and every slice of $\eta'$ is an $\bigO(q')$-approximation for $A$.
  By definition, each slice of $\zeta$ is the union of a slice of $\eta$ and a slice of $\eta'$.
  The corresponding approximations on the constituent slices can be combined into an approximation of which the slice of $\zeta$ at hand is the domain.
  Now, let $p$ be the polynomial that maps $n$ to $q(n) + q'(n)$.
  Regarding the running time of the composite approximation, we find that it can be kept in $\bigO(p)$.
  Because $p$ is independent of the specific slice of $\zeta$, this puts $\zeta$ in $\calF_{(A, \clnu{FPT})}$.
\end{proof}

The above theorem allows us to think of $\quasilenu$ as an inverse, nonuniform, ranking of how powerful parameterizations are.
Suppose that a set is in \clXnu{P} or \clnu{FPT} with parameterizations $\eta$ and $\eta'$.
The \emph{upward closed} property entails that the set is also in the parameterized complexity class with all parameterizations less powerful than $\eta$ or $\eta'$.
The inclusion of greatest lower bounds means that $\eta$ and $\eta'$ can be combined into a parameterization that is more powerful than either one of them.

\subsubsection{Uniform Complexity Classes}
While the absence of uniformity constraints may be accommodating to certain proof methods, uniform parameterized complexity is more practically relevant.
Luckily, we can also characterize the structure of parameterizations relative to uniform parameterized complexity.
\begin{theorem}
\label{thm:lattice}
  Ordered by $\quasile$, the equivalence classes in $\calL_\clX{P}$ and $\calL_\cl{FPT}$ form bounded distributive lattices.
\end{theorem}
\begin{proof}
  Proving the current theorem requires attention to two aspects that did not play a role in proving  Theorem~\ref{thm:nulattice}.
  First, the criterion for membership in $\calL_\clX{P}$ or $\calL_\cl{FPT}$ is more elaborate than that for membership in their nonuniform counterparts.
  Second, whenever a bound on the values of a gap function between two parameterizations is employed, that bound must now be computable.
  Despite these extra elements to consider, a proof of the current theorem may proceed largely along the same lines as the proof of Theorem~\ref{thm:nulattice}.
  For starters, a proof concerning $\calL_\clX{P}$ is again subsumed in one concerning $\calL_\cl{FPT}$ and we shall therefore present only the latter.

  \paragraph{$\calL_\cl{FPT}$ is bounded.}
  The parameterization $(\binary^+)_{k \in \binary^+}$ consisting of full slices is also a least element of $\calL_\cl{FPT}$ with respect to $\quasile$.
  Any constant function acts as a computable upper bound to the gap from this parameterization to any other.
  As, for instance, the empty set is put in \cl{FPT} by this parameterization, it is indeed a member of $\calL_\cl{FPT}$.

  In order for a parameterization to be a greatest element of $\calL_\cl{FPT}$, it is no longer sufficient for its slices to be finite.
  Specifically, the construction of a computable upper bound to the gap function requires knowing the number of instances in each slice.
  With the length parameterization, $(\{x \st \length{x} \le \asNat(k)\})_{k \in \binary^+}$, this extra requirement is satisfied.
  That it is a member of $\calL_\cl{FPT}$ is, again, witnessed by the empty set, which is put in \cl{FPT} by it.
  We claim that the gap from any parameterization to the length parameterization is computable.
  Observe that, for every parameterization $\eta$ that corresponds to a direct parameterized procedure, the minimization function $\mu_\eta$ is computable.
  This is useful, because all parameterizations in $\calL_\cl{FPT}$ correspond to direct parameterized procedures.
  As a result, the gap from a parameterization $\eta \in \calL_\cl{FPT}$ to the length parameterization is also computable.
  For every argument, the gap is simply the maximum of a known finite set of computable values.

  \paragraph{$\calL_\cl{FPT}$ contains greatest lower bounds.}
  A parameterization is in $\calL_\cl{FPT}$ when it corresponds to a direct parameterized procedure that meets the requirements of Definition~\ref{def:fpt}.
  A greatest lower bound for two parameterizations $\eta$ and $\eta'$ in $\calL_\cl{FPT}$ can be constructed uniformly via such parameterized procedures.
  Let $\phi$ and $\phi'$ be procedures associated with $\eta$ and $\eta'$.
  Although $\phi$ and $\phi'$ need not converge to the same set, we can combine them in a parameterized procedure $\psi$ defined by
  \begin{equation*}
    \psi(x, k) \deq \begin{cases}
      \bits{?}	&\text{if $\phi(x, k) = \bits{?}$ and $\phi'(x, k) = \bits{?}$} \\
      \bits{0}	&\text{otherwise.}
    \end{cases}
  \end{equation*}
  This parameterized procedure converges to the empty set and witnesses that the empty set is in \cl{FPT}.
  The parameterization corresponding to $\psi$ is the greatest lower bound $\zeta$ as constructed in the proof of Theorem~\ref{thm:nulattice}.

  What remains is to show that the gap from $\zeta$ to $\eta$ and $\eta'$ can be bounded by a computable function.
  To this end, observe that, for any $k$, an instance is in $\eta_k$ or $\eta'_k$ only if it is in $\zeta_{\pair{k}{k}}$.
  Combined with the bound $\length{\pair{k}{k}} \le 4\length{k}$, it follows that, for every $n$, both $\gap_{\zeta, \eta}(n)$ and $\gap_{\zeta, \eta'}(n)$ are bounded by $4n$.

  \paragraph{$\calL_\cl{FPT}$ contains least upper bounds.}
  For the presence of least upper bounds, the bounds on the gap in the proof of Theorem~\ref{thm:nulattice} are already computable.
  The only thing required for the proof to carry over to the uniform case is to show that the parameterization $\zeta$ as constructed is a member of $\calL_\cl{FPT}$.
  Therefore, we turn, like before, to the parameterized procedures $\phi$ and $\phi'$ associated with parameterizations $\eta$ and $\eta'$ in $\calL_\cl{FPT}$.
  The difference with the case for greatest lower bounds is that our derived  procedure now returns \bits{?} if $\phi$ \emph{or} $\phi'$ produces \bits{?}.
  Corresponding to this parameterized procedure, we find the parameterization $\zeta$, which is hence a member of $\calL_\cl{FPT}$.

  \paragraph{$\calL_\cl{FPT}$ is distributive.}
  The proof of distributivity of the nonuniform lattice is concerned only with obtaining bounds on the gap function.
  As all these bounds in the proof of Theorem~\ref{thm:nulattice} are computable, the proof works equally well for distributivity of the uniform lattice.
\end{proof}

Also within the uniform lattices, filters are defined by sets.
Contrary to the nonuniform case, the sets need to be decidable for otherwise the induced collection of parameterizations is empty.
This is implied by Theorem~\ref{thm:decidable}, which says that direct parameterized procedures can only converge to decidable sets.
\begin{theorem}
\label{thm:filter}
  Let $A$ be a decidable set.
  Ordered by $\quasile$, the collections $\calF_{(A, \clX{P})}$ and $\calF_{(A, \cl{FPT})}$ are filters in $\calL_\clX{P}$ and $\calL_\cl{FPT}$, respectively.
\end{theorem}
\begin{proof}
  This proof too is presented for \cl{FPT}, but works just as well for \clX{P}.

  \paragraph{$\calF_{(A, \cl{FPT})}$ is nonempty.}
  The direct parameterized procedure of which the existence is asserted by Theorem~\ref{thm:slow_decidable} witnesses that $\calF_{(A, \cl{FPT})}$ is nonempty.

  \paragraph{$\calF_{(A, \cl{FPT})}$ is upward closed.}
  Let $\eta$ be a parameterization corresponding to a parameterized procedure $\phi$ that converges to $A$ and meets the requirements of Definition~\ref{def:fpt}.
  We shall show that all parameterizations in $\calL_\cl{FPT}$ that are above $\eta$ according to $\quasile$ are also in $\calF_{(A, \cl{FPT})}$.
  Suppose a parameterization $\eta'$ and a computable function $g$ such that, for all $n$, we have $\gap_{\eta, \eta'}(n) \le g(n)$ are given.
  This parameterization $\eta'$ corresponds to some parameterized procedure $\phi'$ that meets the requirements of Definition~\ref{def:fpt}.
  Consider a parameterized procedure $\psi$ that does the following on input $(x, k)$.
  \begin{codelisting}
  \item
    \code{If} $\phi'(x, k)$ yields \bits{?}, we conclude that $x$ is not in $\eta'_k$ and \code{return} \bits{?}.
  \item\label{code:filter:upward:loop}
    There must be a $k'$ of length at most $g(\length{k})$ such that $x$ is in $\eta_k$.
    Therefore, we are sure to return $A(x)$ by proceeding as follows \code{for all} $k'$ of length at most $g(\length{k})$.
    \begin{codelisting}
    \item \code{If} $\phi(x, k')$ does not yield \bits{?}, \code{return} its output.
    \end{codelisting}
  \end{codelisting}

  To prove that $\psi$ meets the requirements of Definition~\ref{def:fpt}, it suffices to show that step~\ref{code:filter:upward:loop} does.
  Crucially, the parameter dependence of the running time of this loop should be bounded by a computable function.
  If $f$ is a computable function bounding the parameter dependence of the running time of $\phi$, the parameter dependence of the loop can be bounded by
  \begin{equation*}
    \sum_{\mathclap{k'\colon \length{k'} \le g(\length{k})}} f(k').
  \end{equation*}
  This bound is computable, and since $\psi$ converges to $A$ by construction, we find that $\eta'$ puts $A$ in \cl{FPT}.
  As $\eta'$ was an arbitrary parameterization above $\eta$ in the uniform order on parameterizations, we conclude that $\calF_{(A, \cl{FPT})}$ is upward closed.

  \paragraph{$\calF_{(A, \cl{FPT})}$ contains greatest lower bounds.}
  Every parameterization in $\calF_{(A, \cl{FPT})}$ corresponds to a direct parameterized procedure that converges to $A$ and meets the requirements of Definition~\ref{def:fpt}.
  Thus, parameterizations $\eta$ and $\eta'$ taken from $\calF_{A, \cl{FPT})}$ are associated to parameterized procedures $\phi$ and $\phi'$, both converging to $A$.
  These procedures can be combined like in the construction of a greatest lower bound in the proof of Theorem~\ref{thm:lattice}.
  However, instead of returning \bits{0}, we let the constructed procedure return the output of $\phi$ or $\phi'$ that is not \bits{?}.
  If both do not output \bits{?}, their outputs are the same, since both parameterized procedures converge to $A$.
  Thus we obtain a direct parameterized procedure that converges to $A$ on a greatest lower bound of $\eta$ and $\eta'$, and meets the requirements of Definition~\ref{def:fpt}.
  Because of this, we may conclude that $\calF_{(A, \cl{FPT})}$ contains all greatest lower bounds.
\end{proof}

By the above theorem, the uniform order $\quasile$ counts as an inverse ranking of how powerful parameterizations are in a very practical and strong sense.
When a parameterization is strictly below another, it signifies that the convergence behavior of this parameterization is an improvement over that of the other.
This improvement is of a much stronger kind than the improvements made in typical algorithms races.
These races seek improvements within a single parameterization \parencite{komusiewicz2012new,fellows2013towards}.
In this light, the best parameterizations are those that are below most others.
A set $A$ admits an optimal parameterization with respect to, say, \cl{FPT} if the filter $\calF_{(A, \cl{FPT})}$ is principal.

\subsection{Optimal Nonuniform Parameterizations}
With respect to \clXnu{P}, all sets admit optimal parameterizations.
\begin{theorem}
\label{thm:nuxpprincipal}
  For any set $A$, the filter $\calF_{(A, \clXnu{P})}$ is principal.
\end{theorem}
\begin{proof}
  Let $S_1, S_2, S_3, \ldots$ be an enumeration of the polynomial segments of $A$.
  Consider the parameterization $\eta$ given by
  \begin{equation*}
    \eta_k \deq \bigcup_{\mathclap{i \le \asNat(k)}} S_i.
  \end{equation*}
  By definition, $A$ is in \clXnu{P} with $\eta$ and by construction, $\eta$ is a least element in $\calF_{(A, \clXnu{P})}$.
\end{proof}

We shall call a least element in the filter corresponding to some set a \defkey{principal parameterization} for that set.
Theorem~\ref{thm:nuxpprincipal} shows that all sets have principal parameterizations with respect to \clXnu{P}.
However, this is not a given for arbitrary parameterized complexity classes.
When they exist, principal parameterizations provide insight into some of the computational complexity of a set.
For instance, there is a one-to-one correspondence between the imix property of a principal parameterization with respect to \clXnu{P} and the levelability of a set.
This correspondence arises as a consequence of Theorem~\ref{thm:nuxpprincipal}.
\begin{corollary}
  A set is \levelable{\cl{P}} (\immune{\cl{P}}) if and only if a principal parameterization in the induced nonuniform filter with respect to \clXnu{P} has (does not have) imix.
\end{corollary}

Note that the filter induced by a \cl{P}-bi-immune set consists of a single equivalence class of parameterizations.
Namely it consists only of the class of parameterizations $\eta$ where for every parameter value $k$ the set $\eta_k$ is finite.
For filters with respect to \clnu{FPT}, this is no different the filters induced by \immune{\cl{P}} sets are again principal.
\begin{theorem}
\label{thm:nufptprincipal}
  For any set $A$ that is \immune{\cl{P}}, the filter $\calF_{(A, \clnu{FPT})}$ is principal.
\end{theorem}
\begin{proof}
  By definition of being \immune{\cl{P}}, $A$ has a maximal polynomial segment $S$.
  For some polynomial $p$, this polynomial segment $S$ is also an $\bigO(p)$-segment.
  Thus there exists a parameterization that has $S$ as one of its slices and with which $A$ is in \clnu{FPT}.
  This parameterization is equivalent to any parameterization below it in $\calF_{(A, \clnu{FPT})}$, hence it is a principal parameterization for $A$.
\end{proof}

Certainly, a principal parameterization with respect to \clnu{FPT} for any \immune{\cl{P}} set does not have imix.
On the other hand, if the filter with respect to \clnu{FPT} were to be principal for any \levelable{\cl{P}} set, a principal parameterization has to have imix.
The \levelable{\cl{P}} property is, however, indifferent to the degree of the polynomials involved in the polynomial segments of a set.
This makes an investigation of filters with respect to \clnu{FPT} induced by \levelable{\cl{P}} sets difficult.
We need a variant that is sensitive to the degree of the polynomials related to polynomial segments of a set.
\begin{definition}
  A set is \defkeyat{omni-levelable!for FPT@for \cl{FPT}}{\levelable{\cl{FPT}}} if there is a $c$ such that for all polynomials $p$ of degree at least $c$ the set has no maximal $\bigO(p)$-segment.
\end{definition}

Likewise, we could define a set being \immune{\cl{FPT}} as it having a maximal $\bigO(p)$-segment for all polynomials $p$ of sufficiently high degree.
In contrast to the general, degree independent definitions, these two classifications do not exhaust all sets.
Sets may exist that are neither \levelable{\cl{FPT}}, nor \immune{\cl{FPT}}.
Such sets would, however, be \levelable{\cl{P}}.
Optimal parameterizations with respect to \clnu{FPT} do not exist for sets that are \levelable{\cl{P}}, yet not \levelable{\cl{FPT}}.
\begin{theorem}
\label{thm:nufptnonprincipal}
  For any set $A$ that is \levelable{\cl{P}} and not \levelable{\cl{FPT}}, the filter $\calF_{(A, \clnu{FPT})}$ is nonprincipal.
\end{theorem}
\begin{proof}
  In case $A$ is \levelable{\cl{P}} and not \levelable{\clnu{FPT}}, there is an infinite set of polynomials $\{p_1, p_2, p_3, \ldots\}$ such that, for all $i$,
  \begin{itemize}
  \item $A$ has a maximal $\bigO(p_i)$ segment, and
  \item if $S_i$ is a maximal $\bigO(p_i)$-segment of $A$ and $S_{i + 1}$ is a maximal $\bigO(p_{i + 1})$-segment of $A$, then $S_{i + 1} \backslash S_i$ is infinite.
  \end{itemize}
  Any parameterization with which $A$ is in \clnu{FPT} can only contain finitely many maximal $\bigO(p_i)$-segments of $A$ as its slices.
  Hence $\calF_{(A, \clnu{FPT})}$ cannot be principal.
\end{proof}

Together, Theorem~\ref{thm:nufptprincipal} and Theorem~\ref{thm:nufptnonprincipal} provide an overview of what sets have optimal parameterizations with respect to \clnu{FPT}.
This overview is depicted in Figure~\ref{fig:nuprincipal}.
\begin{figure}
  \centering
  \begin{tabular}{|cccc|}
    \multicolumn{2}{|c|}{\immune{\cl{P}}}	& \multicolumn{2}{|c|}{\levelable{\cl{P}}} \\
    \multicolumn{2}{|c|}{}	&	& \multicolumn{1}{|c|}{\footnotesize{\levelable{\cl{FPT}}}} \\
    \hline
    \multicolumn{2}{|c|}{\emph{principal}}	& \multicolumn{1}{|c|}{\emph{nonprincipal}}	& open \\
  \end{tabular}
  \caption{
    The universe of sets, represented by the horizontal line, can be divided according to levelability with respect to \cl{P}.
    Additionally, a subclass of the \levelable{\cl{P}} sets is \levelable{\cl{FPT}}.
    Below the horizontal line, principality of the filter with respect to \clnu{FPT} is indicated.
    For \levelable{\cl{FPT}} sets, no results are available.
  }
  \label{fig:nuprincipal}
\end{figure}

Of course, Theorem~\ref{thm:nufptnonprincipal} is only meaningful if there are \levelable{\cl{P}} sets that are not \levelable{\cl{FPT}}.
This is the case.
\begin{theorem}
  There are \levelable{\cl{P}} sets that are not \levelable{\cl{FPT}}.
\end{theorem}
\begin{proof}
  We shall prove the theorem by constructing a \levelable{\cl{P}} set that has a maximal $\bigO(p)$-segment for infinitely many polynomials $p$ of distinct degrees.
  For this, let $\phi_1, \phi_2, \phi_3, \ldots$ be an effective enumeration of all partial procedures.

  Consider a recursive procedure for deciding a set $A$ that, on input $\pair{c}{x}$, proceeds as follows.
  \begin{codelisting}
  \item
    We determine a set $I$ of indices of procedures that are consistent with an initial segment of $A$:
    \begin{codelisting}
    \item
      \code{Initialize} $I$ to $\{1, 2, 3, \ldots, \length{\pair{c}{x}}\}$.
    \item
      \code{For each} pair $\pair{d}{y}$ that satisfies $d \le c$ and $\length{\pair{d}{y}} \le \log \length{\pair{c}{x}}$ we remove those indices from $I$ that are not consistent with $A(\pair{d}{y})$:
      \begin{codelisting}
      \item
        Recursively \code{compute} $A(\pair{d}{y})$.
      \item
        \code{For each} index $i$ in $I$:
        \begin{codelisting}
        \item
          \code{Simulate} up to $\length{\pair{d}{y}}^{3c}$ steps of $\phi_i$ on input $\pair{d}{y}$.
        \item
          \code{If} $\phi_i$ was simulated to completion and was inconsistent with $A(\pair{d}{y})$ in the sense that we have $\phi_i(\pair{d}{y}) = \bits{1} - A(\pair{d}{y})$,
          \\\-\quad \code{remove} $i$ from $I$.
        \end{codelisting}
      \end{codelisting}
    \end{codelisting}
  \item
    We try to make an index in $I$ inconsistent with $A$:
    \begin{codelisting}
    \item
      \code{For each} index $i$ in $I$:
      \begin{codelisting}
      \item
        \code{Simulate} up to $\length{\pair{c}{x}}^{3c}$ steps of $\phi_i$ on input $\pair{c}{x}$.
      \item
        \code{If} $\phi_i$ was simulated to completion and we have $\phi_i(\pair{c}{x}) \in \{\bits{1}, \bits{0}\}$,
        \\\-\quad \code{return} $\bits{1} - \phi_i(\pair{c}{x})$.
      \end{codelisting}
    \item
      Else, as no procedure could be made inconsistent, \code{return} \bits{0}.
    \end{codelisting}
  \end{codelisting}

  The first stage of this procedure performs at most $\length{\pair{c}{x}}^2$ simulations of computations, each of at most $\log(\length{\pair{c}{x}})^{3c}$ steps.
  Besides these simulations, this stage computes an initial segment of $A$ to test against.
  This segment is computed recursively and the recursion depth is bounded by the iterated logarithm of $\length{\pair{c}{x}}$.
  By using dynamic programming, the time required to compute the segment is insignificant with respect to the total running time of the entire procedure.

  The second stage of the procedure requires the simulation of at most $\length{\pair{c}{x}}$ computations, each of at most $\length{\pair{c}{x}}^{3c}$ steps.
  Efficient simulation \parencite{arora2009computational} limits the overhead of simulation to a logarithmic factor.
  As a function of the input $\pair{c}{x}$, the number of steps spent in the second stage is thus in $\bigO(\length{\pair{c}{x}} \cdot \length{\pair{c}{x}}^{3c} \cdot \log(\length{\pair{c}{x}}^{3c}))$.
  This puts the number of steps taken by the entire procedure in $\bigO(\length{\pair{c}{x}}^{3c + 2})$.
  Note that the running time of the procedure is not polynomial in the length of the input, $\length{\pair{c}{x}}$, as $c$ appears in the exponent.

  Given a constant $c$, let $p_c$ be the polynomial defined by $p(n) \deq n^{3c + 2}$.
  For any fixed $c$, the set $\{\pair{d}{x} \st d \le c \reland x \in \binary^+\}$ is an $\bigO(p_c)$-segment of $A$.
  It is not a maximal polynomial segment, as for larger values of $c$ infinitely many elements are introduced in the corresponding sets.
  However, we claim that it is a maximal $\bigO(p_c)$-segment of $A$ and thus that $A$ is \levelable{\cl{P}}, yet not \levelable{\cl{FPT}}.
  Suppose towards a contradiction that there is an infinite $\bigO(p_c)$-segment $S \subseteq \{\pair{w}{x} \st w > c \reland x \in \binary^+\}$ for $A$.
  Let $i$ be an index of an $\bigO(p_c)$-approximation for $A$ with domain $S$.
  Allmost all $\pair{w}{x} \in S$ will be so that $\length{\pair{w}{x}} \ge i$.
  Therefore, for allmost all inputs to our procedure, $i$ will be included in $I$ when the procedure enters its second stage.
  Because we have $w > c$, we have $3w > 3c + 2$.
  This means that any function in $\bigO(p_c)$ will eventually be dominated by the polynomial mapping $n$ to $n^{3w}$, for any value of $w$ that occurs in $S$.
  Indeed, this is the reason for the constant $3$ in the exponent of the time bounds in our decision procedure for $A$.
  Now, for almost all elements of $S$, the second stage of our procedure does one of two things.
  Either it invalidates $i$ as the index of an $\bigO(p_c)$-approximation for $A$, or it invalidates an index smaller than $i$.
  The latter of these possibilities can happen at most $i - 1$ times, so, since $S$ was assumed to be infinite, eventually $i$ must be invalidated.
  This contradicts our choice of $i$ as the index of an $\bigO(p_c)$-approximation for $A$ with domain $S$.
  We conclude that there is no infinite $\bigO(p_c)$-segment $S \subseteq \{\pair{w}{x} \st w > c \reland x \in \binary^+\}$ for $A$.
\end{proof}

For completeness, we shall also show the existence of \levelable{\cl{FPT}} sets.
Our proof revolves around length increasing reductions.
Our notion of a \emph{reduction} is that of a membership preserving polynomial time computable function, in other words, that of a Karp reduction.
\begin{theorem}
  Every set outside \cl{P} from which there is a linearly length increasing reduction to itself is \levelable{\cl{FPT}}.
\end{theorem}
\begin{proof}
  Let $A$ be a set outside \cl{P} and $f$ a linearly length increasing reduction from $A$ to itself.
  Suppose that $A$ is not \levelable{\cl{FPT}} and, for some polynomial $p$ of degree $c$, has a maximal $\bigO(p)$-segment $S$.
  We may assume that, for some polynomial $q$ of degree $c - 1$, it is possible to compute $f$ in time $\bigO(q)$.
  The sets
  \begin{align*}
    S' &\deq \{x \st x \notin S \reland f(x) \in S\}, \\
    S_x &\deq \{x, f(x), f(f(x)), \ldots\}
  \end{align*}
  are, by nature of $f$, also $\bigO(p)$-segments of $A$.
  For $S'$ this requires the linear bound on the length of the output of $f$, where for $S_x$ this requires that $f$ is length increasing.
  Furthermore, $S'$ satisfies $S \cap S' = \emptyset$.

  By the assumed maximality of $S$, for every $x$ there are only finitely many elements in the set $S \backslash S_x$.
  However, since $A$ is not in \cl{P}, there are infinitely many $x$ outside $S$ and for each of these the set $S_x$ intersects $S'$.
  Hence $S'$ is infinite, contradicting the maximality of $S$.
\end{proof}

The existence of \levelable{\cl{FPT}} sets now follows from the existence of sets outside \cl{P} that have a linearly length increasing reduction to itself.
\begin{lemma}
  There are sets outside \cl{P} that have a linearly length increasing reduction to itself.
\end{lemma}
\begin{proof}
  Let $X$ be a set outside \cl{P} and consider its cylindrification
  \begin{equation*}
    A \deq \{\pair{x}{y} \st x \in X \reland y \in \binary^+\}.
  \end{equation*}
  Note that $A$ too is not in \cl{P}.
  Our pairing function is such that the function $f$ defined by
  \begin{equation*}
    f(\pair{x}{y}) \deq \pair{x}{\bits{0}y}
  \end{equation*}
  is a linearly length increasing reduction from $A$ to itself.
\end{proof}

It should be noted that certain padding functions give rise to linearly length increasing reductions.
There are padding functions, $\pad$, for which there exists a constant $c$ such that for all $x, y$ we have
\begin{equation*}
  \frac{1}{c} \cdot (\length{x} + \length{y}) \le \length{\pad(x, y)} \le c \cdot (\length{x} + \length{y}).
\end{equation*}
Padding functions that meet this enhanced honesty criterion can be turned into linearly length increasing reductions by mapping $x$ to $\pad(x, \bits{0}^{c \cdot \length{x}})$.

It was shown \parencite{young1983some} that every honest, one-one, polynomially computable function is the productive function for the complement of some $k$-creative set.
In particular, this means that there are $k$-creative sets where the influence of the associated productive function on the length of its input is linear.
Such functions can again be turned into linearly length increasing reductions.
Hence, assuming we have $\cl{P} \ne \cl{NP}$, there are also $k$-creative sets that are \levelable{\cl{FPT}}.

\subsection{Optimal Uniform Parameterizations}
The proof of Theorem~\ref{thm:nuxpprincipal} fails for filters with respect to the uniform \clX{P}.
First and foremost, the enumeration of polynomial segments is in general not effective by Rice's theorem.
Additionally, the existence of a procedure that meets the requirements of Definition~\ref{def:xp} is not guaranteed.
Lastly, being a principal parameterization with respect to a uniform complexity class requires a bound on the gap to any other parameterization to be computable.
When focussing only on parameterizations that \emph{provably} put a set in \clX{P}, we regain something akin to Theorem~\ref{thm:nuxpprincipal}.
Here, provability refers to being deducible in some fixed formal system \parencite{kleene1967mathematical}.
\begin{theorem}
\label{thm:xpprincipal}
  Given any formal proof system, for any set $A$ there is a least parameterization among those provably, via a witnessing parameterized procedure and parameter dependence, putting $A$ in \clX{P}.
\end{theorem}
\begin{proof}
  Let $\mathfrak{F}$ be the fixed deductive system.
  A form of universal search, along the lines of \textcite{hutter2002fastest}, through polynomial approximations is possible in the parameterized setting.
  To this end, let $\phi_1, \phi_2, \phi_3, \ldots$ be an effective enumeration of all partial procedures and $p_1, p_2, p_3, \ldots$ an effective enumeration of all polynomials.
  Consider the parameterized procedure converging to $A$ that, on input $(x, k)$, does the following.
  \begin{codelisting}
  \item
    We construct a finite set $M$ of polynomial approximations for $A$:
    \begin{codelisting}
    \item
      \code{Initialize} $M$ to the empty set.
    \item
      \code{For each} combination of a proof $\mathfrak{p}$ in $\mathfrak{F}$, an index $i$ of a parameterized procedure, and an index $j$ of a polynomial, all of length at most $\asNat(k)$:
      \begin{codelisting}
      \item\label{alg:xpprincipal:approximations}
        \code{If} $\mathfrak{p}$ proves that $\phi_i$ is a $p_j$-approximation of $A$,
        \\\-\quad \code{Add} $i$ to $M$.
      \end{codelisting}
    \end{codelisting}
  \item
    We decide on membership of $x$ if any of the polynomial approximations in $M$ does:
    \begin{codelisting}
    \item
      \code{For each} index $i$ in $M$:
      \begin{codelisting}
      \item
        \code{Compute} $\phi_i(x)$.
      \item
        \code{If} $\phi_i(x) \in \{\bits{1}, \bits{0}\}$, \code{return} $\phi_i(x)$.
      \end{codelisting}
    \item
      Else, as none of the polynomial approximation decides on membership of $x$, \code{return} \bits{?}.
    \end{codelisting}
  \end{codelisting}

  We shall first show that $A$ is in \clX{P} with the parameterization corresponding to this procedure.
  After that, we shall show that the gap from that parameterization to any other with which $A$ is provably in \clX{P} can be bounded by a computable function.

  For all parameter values $k$, the set $M$ is finite throughout the execution of this parameterized procedure.
  As a result, the first stage of this procedure takes a finite number of steps and the second stage can be executed in polynomial time, for every fixed $k$.
  Moreover, we claim that the parameter dependence is computable.
  For the first stage, the parameter dependence can be computed by simply performing the prescribed computation and clocking the number of steps taken.
  Observe that we can also keep keep track of all the polynomials $p_j$ associated with indices $i$ added to $M$ in step \ref{alg:xpprincipal:approximations}.
  Doing so, we can upper bound the running time of the second stage by the sum of all these polynomials.
  Because this bound depends solely on the parameter value, we find that we can construct a computable function $f$ as required by Definition~\ref{def:xp}.
  Thus, $A$ is in \clX{P} with the parameterization corresponding to the parameterized procedure above.

  Let $\psi$ be a direct parameterized procedure converging to $A$, and $f$ a computable function.
  Suppose $\psi$ and $f$ are witnesses in a proof, $\mathfrak{p}$, of the fact that $A$ is in \clX{P} with the parameterization corresponding to $\psi$.
  Denote the parameterization corresponding to the $\psi$ by $\zeta$ and the parameterization corresponding to the parameterized procedure we constructed by $\eta$.
  We claim that there is a computable function $g$ such that, for all $k'$, slice $\zeta_{k'}$ is included in slice $\eta_{g(k')}$.
  This entails the desired relationship $\eta \quasile \zeta$.
  Given $\psi$ and $k$ we can effectively come up with an index $i$ such that $\phi_i$ corresponds to the partial application of $\psi$ to $k'$.
  Correspondingly, using $f$ we can come up with an index $j$ such that $\phi_i$ is a $p_j$-approximation of $A$.
  A proof of this, $\mathfrak{p}'$, can be adapted from $\mathfrak{p}$.
  Now, consider the function defined by
  \begin{equation*}
    g(k') \deq \asStr(\max\{\length{i}, \length{j}, \length{\mathfrak{p}'}\}).
  \end{equation*}
  As $i$, $j$, and $\mathfrak{p}'$ were derived effectively from $\psi$, $f$, $\mathfrak{p}$, and $k'$, this is a computable function of $k'$ for any fixed $\psi$, $f$, and $\mathfrak{p}$.
  For any $k'$, the first stage of the above parameterized procedure includes the corresponding $i$ in the set $M$ when we have $k = g(k')$.
  The second stage of the procedure then ensures that slice $\zeta_{k'}$ is included in slice $\eta_k = \eta_{g(k')}$, as we set out to prove.
\end{proof}

Thus, adding a provability requirement offsets the limitations we incurred by moving to a uniform setting.
The provability requirement in Theorem~\ref{thm:xpprincipal} enforces the effectiveness that was not present in the proof of Theorem~\ref{thm:nuxpprincipal}.
With respect to \cl{FPT}, the proof of Theorem~\ref{thm:nufptprincipal} cannot be reused to show that \immune{\cl{P}} sets induce principal filters.
Specifically, it is no longer sufficient for the gap from one parameterization to another to take on only finite values.
Instead, there must be a uniformly computable bound on the value of the gap function.
Of course, we \emph{can} show that filters with respect to \cl{FPT} are principal for sets in \cl{P}.
\begin{theorem}
\label{thm:fptprincipal}
  For any set $A$ that is in \cl{P}, the filter $\calF_{(A, \cl{FPT})}$ is principal.
\end{theorem}
\begin{proof}
  The parameterization consisting of full slices, $(\binary^+)_{k \in \binary^+}$, is one with which $A$ is in \cl{FPT}.
  Since the class of this parameterization is the least element of the encompassing lattice of parameterizations, the filter must be principal.
\end{proof}

Principal parameterizations with respect to \cl{FPT} for sets in \cl{P} do not have imix.
More broadly, if an \immune{\cl{P}} set has a principal parameterization with respect to \cl{FPT}, this parameterization does not have imix.
This is the extent to which the proof of Theorem~\ref{thm:nufptprincipal} can be applied to the uniform setting.
Conversely, if a set has a principal parameterization with respect to \cl{FPT} that does have imix, the set must be \levelable{\cl{P}}.
\begin{lemma}
\label{lem:fptimix}
  A set that, with respect to \cl{FPT}, has a principal parameterization without imix, is \immune{\cl{P}}.
\end{lemma}
\begin{proof}
  A parameterization $\eta$ without imix has a slice $\eta_k$ such that for all other slices $\eta_{k'}$ the difference $\eta_k \backslash \eta_{k'}$ is finite.
  Of any set $A$ that is in \cl{FPT} with $\eta$, this slice is a polynomial segment.
  If $\eta$ is a principal parameterization for $A$, then $\eta_k$ is even maximal up to finite variations for all slices of all parameterizations in $\calF_{(A, \cl{FPT})}$.
  Because every polynomial segment of $A$ can be turned into a slice of a parameterization with which $A$ is in \cl{FPT}, this means that $A$ must be \immune{\cl{P}}.
\end{proof}

This observation is of use as we turn to principality of filters with respect to \cl{FPT} for \levelable{\cl{P}} sets.
While the proof of Theorem~\ref{thm:nufptnonprincipal} works for the uniform setting as well, the reach of the result can be extended.
We shall do so in a way that has a clear kinship to the diagonal argument used in the proof of the time hierarchy theorem of \textcite{hartmanis1965computational}.
Though, where the time hierarchy theorem constitutes a hierarchy of \emph{sets}, our theorem is about a hierarchy of \emph{parameterized procedures}.
\begin{theorem}
\label{thm:fptnonprincipal}
  For any set $A$ that is \levelable{\cl{P}}, the filter $\calF_{(A, \cl{FPT})}$ is nonprincipal.
\end{theorem}
\begin{proof}
  We present a proof by contradiction, assuming $\eta$ is a principal parameterization with respect to \cl{FPT} for $A$.
  A contradiction is arrived at by the construction of a parameterization $\zeta$ with which $A$ is in \cl{FPT} and for which we have $\eta \not\quasile \zeta$.
  Let $\phi$ be a direct parameterized procedure and $p$ a polynomial witnessing that $A$ is in \cl{FPT} with $\eta$ in accordance with Definition~\ref{def:fpt}.
  Consider a direct parameterized procedure that converges to $A$ and proceeds as follows on input $(x, k)$.
  \begin{codelisting}
  \item
    \code{Set a timeout} so that at most $\asNat(k) \cdot \length{x}^2 \cdot p(\length{x})$ steps are spent in total doing the following, \code{for each} $j \in \{1, 2, 3, \ldots\}$
    \begin{codelisting}
    \item
      \code{Compute} $\phi(x, \asStr(j))$.
    \item
      \code{If} $\phi(x, \asStr(j)) \in \{\bits{1}, \bits{0}\}$, \code{return} $\phi(x, \asStr(j))$.
    \end{codelisting}
  \item
    Else, as no decision about $x$ could be reached in time, \code{return} \bits{?}.
  \end{codelisting}

  Let $\zeta$ be the parameterization corresponding to this direct parameterized procedure.
  The dependence on the parameter value in the self-imposed running time bound ensures that $\zeta$ is in fact a parameterization.
  Because the dependence on the length of the instance is a polynomial of which the degree does not depend on the parameter value, $A$ is in \cl{FPT} with $\zeta$.

  It remains to show that we have $\eta \not\quasile \zeta$.
  Observe that our procedure computes values of $\bigO(p)$-approximations for $A$ until it encounters an approximation of which $x$ is in the domain.
  By merit of the $\length{x}^2$ factor in the running time bound, the number of approximations that our procedure can compute increases as a function of $\length{x}$
  In fact, this number increases without a bound for any constant value of $k$.
  From Lemma~\ref{lem:fptimix}, it follows that $\eta$ must have imix.
  Therefore, even for a fixed value of $k$, our procedure is able to decide membership of instances $x$ for which $\mu_\eta(x)$ is arbitrarily high.
  Hence, the gap from $\eta$ to $\zeta$ is always infinite, proving $\eta \not\quasile \zeta$.
\end{proof}

From the perspective of Lemma~\ref{lem:fptimix}, we get from the above theorem that no principal parameterization in a filter of the form $\calF_{(A, \cl{FPT})}$ has imix.
However, parameterizations with imix are the most interesting from an applications point of view.
Thus, we find that sets for which parameterized algorithms are attractive do not admit optimal parameterizations.
When the filter induced by a set is nonprincipal, there is no single structural origin of computational complexity inside the set.
Of such sets, it is not possible to capture the structure responsible for the computational hardness of the set by a parameterization.
Specifically, consider the many structural properties an instance may have.
Some of these properties can be used to decide membership of the instance in a set in a way that defies the computational hardness of the set.
If the set lacks an optimal parameterization, there are infinitely many distinct properties that can be used to that end.

We now have a characterization of which filters with respect to \cl{FPT} are principal and which are not.
In Figure~\ref{fig:principal}, this characterization, a combination of Theorem~\ref{thm:fptprincipal} and Theorem~\ref{thm:fptnonprincipal} is summarized visually.
\begin{figure}
  \centering
  \begin{tabular}{|cccc|}
    \multicolumn{2}{|c|}{\immune{\cl{P}}}	& \multicolumn{2}{|c|}{\levelable{\cl{P}}} \\
    \multicolumn{1}{|c|}{\footnotesize{\cl{P}}}	& \multicolumn{1}{|c|}{}	& \hphantom{\emph{nonprincipal}}	& \hphantom{\footnotesize{\levelable{\cl{FPT}}}} \\
    \hline
    \multicolumn{1}{|c|}{\emph{principal}}	& open	& \multicolumn{2}{|c|}{\emph{nonprincipal}} \\
  \end{tabular}
  \caption{
    The uniform counterpart to Figure~\ref{fig:nuprincipal}.
    For \immune{\cl{P}} sets outside \cl{P}, no results regarding the principality of filters with respect to \cl{FPT} are available.
  }
  \label{fig:principal}
\end{figure}
