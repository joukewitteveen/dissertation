\section{as Limit Computability}
\label{sec:computability}%

Computability theory \parencite{rogers1967theory} is one of the pillars of mathematical logic.
In formal systems, and in particular in proof calculi, questions regarding the decidability of the set of theorems are of central importance \parencite{kleene1967mathematical}.
At the same time, computability theory can be seen as a theory of computational complexity with complete disregard for resource usage.

\todo[inline]{
  This section appears to duplicate contents that should be in the introduction.

  Instead, it should compare and contrast with the material in the introduction, i.e.~with the arithmetical hierarchy.

  Ershov's difference hierarchy stems from ca.~1986, some 25~years after Kleene introduced the arithmetical hierarchy.
  It lives inside $\Delta^0_2$, so it embodies a more detailed notion of complexity (at the cost of not being relevant for too complicated sets).
}
A parameterized study of decidability was started already in 1965 \parencite{putnam1965trial,gold1965limiting}, some thirty years before a parameterized investigation of computational complexity took off.
In this section, we shall recast some of the classical results in an explicitly parameterized framework, noting the new insights thus obtained.
Throughout this treatise, the same parameterized framework will be employed for the analysis of a multitude of complexity notions.
As a result, some of the insights from this section will pertain to notions of complexity different from the various forms of decidability considered here.

\subsubsection{Synopsis}
\todo[inline]{
  A set is of negligible complexity in computability theory if it is decidable.
  Beyond the decidable sets, we find the semidecidable sets.
  However, the story does not end there, as the difference of two semidecidable sets need not be semidecidable.
  This observation has led to a hierarchy where each level consists of sets that can be written as the difference of a fixed number of semidecidable sets.
  A parameterized interpretation says that a set in the $n$th level of this hierarchy occurs as the limit of a parameterized decision procedure that changes its decision at most $n$ times.
  This hierarchy can be further expanded by allowing for the number of alterations to depend on the string of which membership is being decided.

  Findings:
  \\- The two dominant schools of parameterized complexity differ in the class of sets they may consider tractable.
      This difference is of a computability nature.
  \\- The type of reducibility to the halting set (Turing or truth-table) corresponds to the computability of the number of alterations.
  \\- Finite injury constructions can be cleanly expressed in the parameterized framework.
      As an example, we give a proof of the existence of bi-immune \pdash{}selective sets that is much more transparent than the existing proof.

  %broadly, this section shows that parameterized complexity needs to explicitly keep track of parameterizations, the next looks at collections of parameterizations for a fixed set, the section after that looks at fixed collections of parameterizations for varying sets.
}

\subsection{Decidable and Semidecidable Parameterizations}
Arguably the most elementary parameterizations are decidable parameterizations.
With \defkeyat{parameterization!decidable}{decidable}, we mean that there is a procedure that, given some $(x, k) \in \binary^+ \times \binary^+$, decides whether the $k$th slice of the parameterization includes $x$.
From a computability perspective, such parameterizations are fairly uninteresting.
\begin{theorem}
\label{thm:decidable}%
  The following statements about a set $A$ are equivalent.
  \begin{enumerate}[series=enum:decidable]
  \item\label{enum:decidable:decidable}
    $A$ is decidable.
  \item\label{enum:decidable:converging}
    There is a direct parameterized procedure converging to $A$.
  \item\label{enum:decidable:converging_on}
    There is a parameterized procedure converging to $A$ on a decidable parameterization.
  \end{enumerate}
\end{theorem}
\begin{proof}
$\ref{enum:decidable:decidable} \implies \ref{enum:decidable:converging}$.
  We can adapt a decision procedure for $A$ so that it is a direct parameterized procedure converging to $A$.
  This is done by simply ignoring the second input, the parameter value.
  The resulting parameterized procedure correctly decides membership in $A$ regardless of the parameter value.
  Corresponding to this parameterized procedure is the parameterization that consists only of copies of the full set $\binary^+$.

$\ref{enum:decidable:converging} \implies \ref{enum:decidable:converging_on}$.
  More generally, the parameterization corresponding to any direct parameterized procedure that converges to $A$ is decidable.
  To decide whether some given $x$ is in some slice $k$ of a parameterization, we may use the parameterized procedure.
  If the procedure outputs \bits{?}, then slice $k$ of the parameterization does not include $x$, otherwise it does.

$\ref{enum:decidable:converging_on} \implies \ref{enum:decidable:decidable}$.
  Because a parameterization is a cover, for every instance $x$, there is a parameter value $k$ such that the $k$th slice of the parameterization includes $x$.
  Given an instance $x$, such a $k$ can be found by running the decision procedure for the parameterization on $(x, \asStr(1))$, $(x, \asStr(2))$, $(x, \asStr(3))$, and so on, until we find an accepted pair.
  Membership of $x$ in $A$ can then be decided by running the parameterized procedure on the pair thus found.
\end{proof}

In other words, with respect to parameterized procedures, the decidable parameterizations characterize the decidable sets.
Note that the decidability requirement for decidable parameterizations is uniform in the parameter $k$.
A single decision procedure has to be available that works for all values of $k$.
Therefore, the only class we could reasonably call \emph{uniformly fixed-parameter decidable} is the class of decidable problems.
Indeed, this class alone does nothing to warrant our parameterized endeavor.

A naive broadening of our scope does not get us anywhere new.
Calling a parameterization semidecidable when its slices are semidecidable uniformly in the parameter value only leads to an extension of Theorem~\ref{thm:decidable}.
\begin{theorem}
  The statements about a set $A$ of Theorem~\ref{thm:decidable} are equivalent to
  \begin{enumerate}[enum:decidable]
  \item\label{enum:decidable:semidecidable}
    There is a parameterized procedure converging to $A$ on a semidecidable parameterization.
  \end{enumerate}
\end{theorem}
\begin{proof}
$\ref{enum:decidable:converging_on} \implies \ref{enum:decidable:semidecidable}$.
  Since a decidable parameterization is also a semidecidable parameterization, this is immediate.

$\ref{enum:decidable:semidecidable} \implies \ref{enum:decidable:decidable}$.
  As in the proof of ($\ref{enum:decidable:converging_on} \implies \ref{enum:decidable:decidable}$), it suffices to show that a parameter value $k$ can be found such that the $k$th slice of the parameterization includes $x$.
  To see that this is indeed the case, let $\eta$ be a semidecidable parameterization and consider the set
  \begin{equation*}
    \{(x, k) \st x \in \eta_k\}.
  \end{equation*}
  Because $\eta$ is semidecidable uniformly in the parameter value, this set is semidecidable.
  Equivalently, there is a procedure that enumerates this set.
  As in the proof of ($\ref{enum:decidable:converging_on} \implies \ref{enum:decidable:decidable}$), the key insight is now that $\eta$ is a cover.
  This means that we can run the enumeration and simply wait until a pair $(x_i, k_i),\ i \in \bbN$ comes along for which we have $x_i = x$.
  The corresponding parameter value $k_i$ is such that the $k_i$th slice of $\eta$ includes $x$.
\end{proof}

The parameterization corresponding to a direct parameterized procedure is necessarily decidable and cannot be merely semidecidable.
This is a consequence of the fact that parameterized procedures are total.
While they may output \bits{?} instead of either \bits{1} or \bits{0}, they are not permitted to run forever on any input.
It follows that classes of sets in parameterized complexity theory must be closed under taking complements.
For every parameterized procedure that converges to a set on some parameterization, there is a parameterized procedure that converges to the complement of that set on the same parameterization.
The derived procedure simply retains the output if it is \bits{?} and flips it if it is either \bits{1} or \bits{0}.

Furthermore, it is possible to bound the number of steps taken by a parameterized procedure by a function of the parameter value alone.
That is, restricting to bounded parameterized procedures does not affect the class of sets that occur as limits with respect to decidable parameterizations.

\begin{theorem}
\label{thm:slow_decidable}%
  For every decidable set there is a direct parameterized procedure that converges to it and takes a number of steps that only depends on the parameter value.
  On an input $(x, k)$, the procedure terminates within $\bigO(\length{k})$ steps, with the hidden constant not depending on the set at hand.
\end{theorem}
\begin{proof}
  We define a direct parameterized procedure that meets the requirements of the theorem.
  On input $(x, k)$, the procedure simulates the decision procedure for up to $\length{k}$ steps.
  If the simulation terminates, the procedure outputs the result of the simulation.
  Otherwise, it outputs \bits{?}.

  Let $\eta$ be the parameterization associated with the direct parameterized procedure defined above.
  For every two parameter values $k, k'$ such that $\length{k} \le \length{k'}$ holds, we have $\eta_k \subseteq \eta_{k'}$.
  Also, for every instance $x$, the number of steps needed by the decision procedure is finite.
  Therefore, for every instance $x$ there will be a parameter value $k$ such that $x$ is included in slice $\eta_k$.
  Hence, our procedure meets the requirements of the theorem.
\end{proof}

This theorem can be modified for other types of resources and for slower growing bounds in a straightforward way.
When resource bounds are of interest, it may be natural to restrict attention to decidable parameterizations.
Indeed, parameterized complexity theory as developed by \textcite{flum2006parameterized} demand decidability of parameterizations.
We could call such parameterizations \emph{properties}, since it is knowable whether a given parameter value `belongs' to a given instance.
\begin{corollary}
\label{cor:decidable}%
  Let \cl{\itshape C} be a complexity class in a parameterized framework that revolves around properties.
  A set (together with some parameterization) is in~\cl{\itshape C} if and only if it is decidable.
\end{corollary}
This rather informal corollary follows from Theorem~\ref{thm:slow_decidable} because of the way complexity classes in parameterized complexity theory are constructed.
Typically, resource bounds in parameterized complexity theory allow for arbitrary resource usage as a function of the parameter value.
The corollary tells us that parameterized analysis of computational complexity is at least as much about parameterizations as it is about sets.

\subsection{Bounded Undecidability}
Our classifications of a parameterization, decidable and semidecidable, are based on the decidability of elements of the parameterization.
They are horizontal classifications, in the sense that we kept the second component, the parameter value, of the set
\begin{equation*}
  \{(x, k) \st x \in \eta_k\}.
\end{equation*}
fixed.
Granted, by our uniformity constraints we could equally well have kept the first component fixed, but that would not work in a nonuniform theory.

A truly vertical aspect of parameterizations is the point-cofinite property that distinguishes a parameterization from a quasiparameterization.
It allows a characterization of parameterized procedures that converge.
\begin{lemma}
\label{lem:convergent_finite}%
  A parameterized procedure $\phi$ converges to some set on some parameterization if and only if, for all $x$, the sets
  \begin{align*}
    &\{\mathrlap{i}\phantom{k} \st \phi(x, \asStr(i)) \neq \phi(x, \asStr(i + 1))\}\text{ and} \\
    &\{k \st \phi(x, k) = \bits{?}\}
  \end{align*}
  are finite.
\end{lemma}
\begin{proof}
  $\Longrightarrow$.
  Suppose $\phi$ converges to a set $A$ on a parameterization $\eta$.
  For every $(x, k)$, the output of $\phi(x, k)$ can only differ from $A(x)$ when $x$ is not included in $\eta_k$.
  Because $x$ is included in all but finitely many slices of $\eta$, the set of parameter values where $\phi$ changes its decision must hence be finite.
  In particular, $\phi$ cannot output \bits{?} infinitely often.

  $\Longleftarrow$.
  Let $A$ consist of the instances $x$ for which there are infinitely many values of $k$ such that $\phi(x, k)$ outputs \bits{1}.
  Consider the family of sets
  \begin{equation*}
    (\{x \st \phi(x, k) = A(x) \reland \length{x} \le \asNat(k)\})_{k \in \binary^+}.
  \end{equation*}
  Each element of this family is a finite set.
  Because $\phi$ cannot output \bits{?} infinitely often, every instance $x$ is included in all but finitely many elements of this family.
  Hence, this family is a parameterization and $\phi$ converges to $A$ on it.
\end{proof}

Observe that in the right-to-left direction of the above proof, the finiteness of the elements of the family of sets matters.
While a family like
\begin{equation*}
  (\{x \st x \neq k\})_{k \in \binary^+}
\end{equation*}
is a point-cofinite cover of $\binary^+$, it is not directed and therefore not a parameterization.
Making the sets in the family finite the way we did is a fittingly technical solution to this troublesome technicality.

A direct parameterized procedure is one where the only changes on some instance $x$ are from \bits{?} to a `correct' decision or vice versa.
As we have seen, the limit sets of convergent direct parameterized procedures were decidable.
A parameterized foray into the realm of sets that are not necessarily decidable starts with parameterized procedures that are not necessarily direct.
\begin{definition}
  Let $f$ be a function from $\binary^+$ to $\bbN$.
  A parameterized procedure $\phi$ is \defkeyat{parameterized procedure!alternating@$f$-alternating}{$f$"~alternating} if for every instance $x$ there are at most $f(x)$ values of $i$ satisfying
  \begin{equation*}
    \phi(x, \asStr(i)) \neq \phi(x, \asStr(i + 1)).
  \end{equation*}
\end{definition}

Note that in this definition, we have imposed a linear order on the parameter values.
In the presence of a parameterization, this imposed order need not be related to the inclusion order on the slices of the parameterization.

Extending Lemma~\ref{lem:convergent_finite}, the convergent parameterized procedures can be characterized in yet another way.
A parameterized procedure is convergent if it is $f$"~alternating and for every instance $x$ outputs \bits{?} for only finitely many parameter values.
When the function $f$ maps its inputs to the constant $n$, the limit set of an $f$"~alternating parameterized procedure is said to be \emph{weakly $n$"~computably enumerable} \parencite{odifreddi1992classical,epstein1981hierarchies}.
Such sets form the $\Delta^{-1}_n$ levels of a hierarchy known today as the difference hierarchy \parencite{downey2010algorithmic}.\indexkey{difference hierarchy}
The hierarchy originates in the 1960s \parencite{putnam1965trial,gold1965limiting,ershov1968hierarchyi} and is generally attributed to \citeauthor{ershov1968hierarchyi}.
It gets its name from being the closure of the semidecidable sets under an increasing number of applications of the set-theoretic difference operation.
As observed before, classes of limits of parameterized procedures are closed under taking complements.
Therefore, it is no surprise that our parameterized approach favors the $\Delta$ levels of the difference hierarchy.
A complement of a set in a $\Sigma$ level is in the corresponding $\Pi$ level and likewise the other way around.
%TODO: mention that each \Delta level is the intersection of the corresponding \Sigma and \Pi levels

Indeed, the difference hierarchy is strict.
Already in the standard proofs thereof \parencite{arslanov1997degree,ershov1968hierarchyi,putnam1965trial}, we can see a parameterized line of reasoning at work.
We shall give a different and more general proof, exposing a fine-grained structure among the limits of $f$"~alternating sets.
Central to this proof is the ability to bound the running time of an $f$"~alternating parameterized procedure as a function of the parameter value exclusively.
This bounding of the number of steps taken comes at the cost of an increase in the number of alternations of at most $1$.
\begin{theorem}
\label{thm:slow_convergence}%
  Let $\phi$ be an $f$"~alternating parameterized procedure that converges to a set $A$.
  There exists an $(1 + f)$"~alternating parameterized procedure that converges to $A$ and, on every input $(x, k)$, terminates within $\bigO(\length{k})$ steps.
  Here, the hidden constant does not depend on $\phi$.
\end{theorem}
\begin{proof}
  We define a parameterized procedure $\phi'$ with the desired properties as follows.
  On input $(x, k)$, spend $\length{k}$ steps in total on simulating $\phi$ with varying parameter values.
  Specifically, $\phi'$ computes an initial segment of the sequence
  \begin{equation}
  \label{eq:slow_convergence}
    \big(\phi(x, \asStr(1)),\quad \phi(x, \asStr(2)),\quad \phi(x, \asStr(3)), \quad \ldots\big).
  \end{equation}
  After spending $\length{k}$ steps doing so, $\phi'$ yields the output of the last completed simulation, or \bits{?} if no simulation finished.

  Observe that arbitrarily long initial segments of the sequence \eqref{eq:slow_convergence} are computed by $\phi'$.
  The length of the initial segment that can be computed in $\length{k}$ steps is unbounded as a function of the length of $k$.
  By Lemma~\ref{lem:convergent_finite}, this means that $\phi'$ converges to the same set as $\phi$.

  The number of changes that occur for a given $x$ in the sequence \eqref{eq:slow_convergence} is related to $f$.
  After $\phi(x, \asStr(1))$ is included in the sequence, the output of $\phi'$ can change at most $f(x)$ times.
  The first output reproduced from a completed simulation of $\phi$ need not be \bits{?}.
  Therefore, it is possible that the number of changes in the output of $\phi'$ is $1$ higher than $f(x)$.
\end{proof}

Naturally, Theorem~\ref{thm:slow_convergence} can be adapted for resources other than time.
Most importantly, the theorem implies that, when $f$ is computable, it is possible to diagonalize against the $f$"~alternating parameterized procedures.
\begin{theorem}
\label{thm:computable_hierarchy}%
  Let $f$ and $g$ be computable functions such that for infinitely many $x$ we have $f(x) < g(x)$.
  There exists a set that is the limit of a $g$"~alternating parameterized procedure, but not of any $f$"~alternating parameterized procedure.
\end{theorem}
\begin{proof}
  We shall diagonalize against the $f$"~alternating parameterized procedures.
  Because $f$ and $g$ were both assumed to be computable, so is the set $\{x \st f(x) < g(x)\}$.
  For our diagonalization, we may therefore assume that $f$ is less than $g$ everywhere.

  In simulating a parameterized procedure, we can keep track of the number of steps taken.
  Therefore, we can enumerate the parameterized procedures with a running time bounded by the square of the length of the parameter value.
  Moreover, the number of alternations can be counted.
  Thus, we can enumerate the $(1 + f)$"~alternating parameterized procedures with a running time that is at most the square of the length of the parameter value.
  Let $\psi_1, \psi_2, \psi_3, \ldots$ be such an enumeration and consider the parameterized procedure
  \begin{equation*}
    \psi(x, k) \deq \begin{cases}
      \bits{1}	&\text{if $\psi_{\asNat(x)}(x, k) = \bits{0}$}, \\
      \bits{0}	&\text{otherwise}.
    \end{cases}
  \end{equation*}
  This procedure is $(1 + f)$"~alternating and hence it is $g$"~alternating.
  By Theorem~\ref{thm:slow_convergence}, this construction diagonalizes against all $f$"~alternating parameterized procedures.
  Accordingly, there is no $f$"~alternating parameterized procedure that converges to the same set as $\psi$.
\end{proof}

Similar theorems have been published \parencite{epstein1981hierarchies,arslanov1997degree}, but our parameterized framework inspired a notably more elegant proof.

\subsection{Reducibility to the Halting Set}
The undecidable halting set\indexkey{Halt@\pr{Halt}},
\begin{equation*}
  \pr{Halt} \deq \{x \in \binary^+ \st \text{$x$ encodes a procedure that terminates}\},
\end{equation*}
appears as early as is possible in the hierarchy of limits of $f$"~alternating parameterized procedures.
\begin{lemma}
\label{lem:one_halting}%
  There is a $1$"~alternating parameterized procedure that converges to \pr{Halt}.
\end{lemma}
\begin{proof}
  Let us define such a parameterized procedure $\phi$.
  On input $(x, k)$ it simply simulates the procedure encoded by $x$ for up to $\asNat(k)$ steps.
  If the simulation terminates, then $\phi$ outputs \bits{1}, else it outputs \bits{0}.
  Given any instance $x$, for large enough values of $\asNat(k)$ the output of $\phi$ will correspond to membership of $x$ in \pr{Halt}.
  For members of \pr{Halt}, $\phi$ may change its decision once as $\asNat(k)$ becomes large enough.
  Otherwise, $\phi$ outputs \bits{0} and is thence $1$"~alternating.
\end{proof}

As it turns out, the class of limits of $f$"~alternating parameterized procedures where $f$ is computable is closed under truth-table reducibility.
The halting set is complete for this class \parencite{epstein1981hierarchies,arslanov1997degree,downey2010algorithmic}.
\begin{theorem}
\label{thm:truthtable_halt}%
  There is a computable function $f$ and an $f$"~alternating parameterized procedure converging to a set $A$ if and only if $A$ is truth-table reducible to \pr{Halt}.
\end{theorem}
\begin{proof}
  $\Longrightarrow$.
  Let $f$ be a computable function and $\phi$ an $f$"~alternating parameterized procedure converging to a set $A$.
  The idea is to locate, given an instance $x$, the last change in the output of $\phi$ for varying parameter values, using the halting set.
  To this end, let $\psi_x$ be a procedure that takes a number $m$ as input and outputs the $m$th change of $\phi$ on $x$, if it exists.
  Specifically, if $\phi$ changes fewer than $m$ times, $\psi_x$ does not halt on input $m$.
  Otherwise, the output of $\psi_x$ is the $m$th value of $i$ satisfying $\phi(x, \asStr(i)) \neq \phi(x, \asStr(i + 1))$.

  Using at most $f(x)$ queries to \pr{Halt}, we can find the greatest value of $m$ such that $\psi_x(m)$ terminates.
  A truth-table reduction from $A$ to \pr{Halt} may compute this $m$ and output $\phi(x, \asStr(\psi_x(m) + 1))$.
  Thus, the reduction finds the final decision of $\phi$ regarding membership of $x$ in $A$.

  $\Longleftarrow$.
  Let $\psi$ be the $1$"~alternating parameterized procedure converging to \pr{Halt} that we constructed in the proof of Lemma~\ref{lem:one_halting}.
  We shall use $\psi$ to define a computable function $f$ and an $f$"~alternating parameterized procedure $\phi$ converging to $A$.
  On input $(x, k)$, our $\phi$ simulates the reduction from $A$ to \pr{Halt} to obtain a truth-table for $x$.
  Let $y_1, y_2, y_3, \ldots, y_m$ be the queries prescribed by this truth-table.
  Next, $\phi$ evaluates the truth-table using $\psi(y_1, k), \psi(y_2, k), \psi(y_3, k), \ldots, \psi(y_m, k)$ as decisions for the queries.
  For sufficiently large values of $\asNat(k)$, the queries will all be answered correctly.
  Before that, $\phi$ changes its decision for a given $x$ at most $2^m$ times, as the queries cannot be answered in more than that many different ways.
  As $m$ can be computed from $x$, the function $f(x) \deq 2^m$ is computable.
\end{proof}

With regard to the halting set, truth-table reducibility coincides with a weaker variant, aptly named \emph{weak truth-table reducibility} \parencite{odifreddi1992classical,downey2010algorithmic}.
However, it is not as general as Turing reducibility.
Indeed, there is no fastest-growing computable function.
The proof of Theorem~\ref{thm:computable_hierarchy} can be changed so that it also diagonalizes against all computable functions.
Crucially, this depends on the observation that the bound on the number of alternations need not be known for all parameter values.
The instance $x$ of an input $(x, k)$ is thus interpreted as a pair of two procedures.
The first procedure is treated as a parameterized procedure and simulated on $(x, k)$ for some $\length{k}$ steps.
The second procedure is treated as a function bounding the number of alternations made by the first procedure.
It is also simulated on $x$ for $\length{k}$ steps.
In case the second simulation does not finish, we proceed as if it had produced $0$ as output.
For every computable function, computation of the function can be completed if the parameter value is long enough.
Thus, we can diagonalize against all parameterized procedures that, for some computable function $f$, are $f$"~alternating.

This observation shows that the Turing degree of the halting set encompasses more sets than the truth-table degree of the halting set \parencite{epstein1981hierarchies,arslanov1997degree,downey2010algorithmic}.
This gap was closed by \textcite{ershov1968hierarchyii} using an extension of the difference hierarchy beyond the finite levels.
His extension started at $\omega$, the first infinite ordinal.
The sets that are truth-table reducible to the halting set form the $\Delta^{-1}_\omega$ level of the transfinite difference hierarchy.
For larger, yet constructible \parencite{rogers1967theory}, ordinals $\alpha$, the $\Delta^{-1}_\alpha$ level matches the sets that are Turing reducible to the halting set \parencite{ershov1968hierarchyii,epstein1981hierarchies}.
By Post's theorem \parencite{post1948degrees,rogers1967theory}, we can fit this class of sets in the arithmetical hierarchy.\indexkey{arithmetical hierarchy}
A set is Turing reducible to the halting set precisely when it sits at the $\Delta^0_2$ level of the arithmetical hierarchy.

In essence, it was recognized by \textcite{shoenfield1959degrees} that a set is Turing reducible to \pr{Halt} precisely when a parameterized procedure converges to it.
This characterization is known as the \defkey{limit lemma} \parencite{odifreddi1992classical,downey2010algorithmic}.
The original framing of the lemma was around approximations that could change only finitely often for any instance.
In terms of $f$"~alternating parameterized procedures, the characterization foregoes any computability requirements on $f$ and augments Theorem~\ref{thm:truthtable_halt}.
\begin{theorem}
\label{thm:turing_halt}%
  There is a parameterized procedure converging to a set $A$ if and only if $A$ is Turing reducible to \pr{Halt}.
\end{theorem}
\begin{proof}
  $\Longrightarrow$.
  The same approach as in the proof of Theorem~\ref{thm:truthtable_halt} is applicable.
  However, while the required number of queries to \pr{Halt} is still finite, we no longer have a computable upper bound for it.
  Thus, the reduction we obtain need not be a truth-table reduction.
  Instead, we obtain a Turing reduction.

  $\Longleftarrow$.
  The proof of Theorem~\ref{thm:truthtable_halt} does not carry over to the current theorem immediately.
  The reduction from $A$ to \pr{Halt} may be adaptive, so we cannot gather all queries at once.
  Moreover, the reduction may not even terminate if its queries are answered incorrectly.
  While we have no way of answering the queries correctly, a solution is available along the lines of Theorem~\ref{thm:slow_convergence}.

  Let $\psi$ be the parameterized procedure that converges to \pr{Halt} as constructed in the proof of Lemma~\ref{lem:one_halting}.
  We shall use $\psi$ to define a parameterized procedure $\phi$ converging to $A$.
  On input $(x, k)$, our $\phi$ simulates the reduction from $A$ to \pr{Halt} for up to $\length{k}$ steps.
  When the reduction queries membership of some $y$ in \pr{Halt}, the query is answered according to $\psi(y, k)$.
  The steps needed to compute $\psi(y, k)$ are not counted toward the number of steps used in simulating the reduction.
  If the reduction arrives at a decision within the allotted number of steps, the decision is propagated by $\phi$.
  Otherwise, $\phi$ yields \bits{?}.
  For sufficiently long parameter values, this procedure can simulate the reduction to completion and answers all queries in agreement with \pr{Halt}.
\end{proof}

Earlier, we recovered the class of decidable sets in the parameterized context via decidable parameterizations.
Theorem~\ref{thm:truthtable_halt} and Theorem~\ref{thm:turing_halt} can be used in a similar fashion.
By the first, we could call a set a \emph{bounded parameterized limit} precisely when it sits at the $\Delta^{-1}_\omega$ level of the difference hierarchy.
The second suggests to call the sets at the $\Delta^0_2$ level of the arithmetical hierarchy the \emph{parameterized limits}.
Moreover, resource bounds that are independent of the parameter value cannot distill structure inside either of these classes by Theorem~\ref{thm:slow_convergence}.

Like what we found in Corollary~\ref{cor:decidable}, our results connect to a school of parameterized complexity theory.
In the style pioneered by \textcite{downey1999parameterized}, parameterized complexity theory is more permissive than it is with \textcite{flum2006parameterized}.
With \citeauthor{downey1999parameterized}, parameterizations are no longer required to be decidable.
Owing to the lack of an explicitly required decidability, we could call such parameterizations \emph{promises}.
\begin{corollary}
  Let \cl{\itshape C} be a fixed-parameter complexity class in a parameterized framework that revolves around promises.
  A set (together with some parameterization) is in~\cl{\itshape C} if and only if it is in the $\Delta^0_2$ level of the arithmetical hierarchy.\indexkey{arithmetical hierarchy}
\end{corollary}

Comparing this corollary to Corollary~\ref{cor:decidable}, we see that we have gone one step up in the $\Delta$~levels of the arithmetical hierarchy.
\slogan{Properties are related to~$\Delta^0_1$ in the same way as promises are related to~$\Delta^0_2$.}
%TODO: untangle into two slogans (add an example after cor:decidable and phrase as a slogan here).
%TODO: conclude: a parameterized complexity theory based on non-parameterized sets has very limited prospects.
%      technically (using definitions from a later section): if a set is cosidered fixed-parameter tractable when it is in FPT with any parameterization
%Slogan: If parameters are promises, then the fixed-parameter tractable sets are the sets that are semidecidable and have a semidecidable complement.
%TODO: Fix the above slogan. \Delta^0_2 is not the class of semidecidable sets with a semidecidable complement (i.e. the decidable sets), it is one \Delta level higher than that!

\subsection{Subparameterizations}
Let us take a brief excursion from looking at limit sets and examine the structure of parameterizations.
Recall that what distinguishes a parameterization $\eta$ from a quasiparameterization is the property
\begin{equation*}
  \forall x\colon \forall^\infty k\colon x \in \eta_k.
\end{equation*}
We could say that this property is a local property, as it depends on the behavior of the parameterization on individual instances.
Parameterizations most commonly encountered in practice \parencite{niedermeier2006invitation} often enjoy a global variant, namely they satisfy
\begin{equation}
\label{eq:global_cofinite}
  \forall k\colon \forall^\infty k'\colon \eta_k \subseteq \eta_{k'}.
\end{equation}
Indeed, this property is sufficient for a quasiparameterization to be a parameterization, yet it is not necessary.
A parameterization may contain infinitely many incomparable sets, which would prevent it from adhering to \eqref{eq:global_cofinite}.
At the same time, a subset of a quasiparameterization that does obey \eqref{eq:global_cofinite} always exists.
In fact, the subset may be chosen so that it is ordered linearly by inclusion.
\begin{lemma}
\label{lem:cofinal_chain}%
  Every quasiparameterization has a subset that is a linearly ordered parameterization.
  If the inclusion order on the elements of the quasiparameterization is decidable as a function of their indices, such a subset can be found effectively.
\end{lemma}
\begin{proof}
  For the case without decidability constraints, any cofinal chain \parencite{abramsky1994domain} serves as a witness of the lemma.
  Because a quasiparameterization is directed, it contains cofinal chains.

  When the order is decidable, such a chain can be constructed by going through all parameter values and selectively including some of the corresponding slices.
  The slice corresponding to a parameter value is included in the subset of the quasiparameterization if it is a superset of the last slice that was added.
  To the $i$th slice thus included, we associate the string representation of $i$ as a parameter value.
  This procedure is effective and by construction the resulting subset is ordered linearly.
  It remains to show that it is cofinal.
  If it was not, than there would be a slice of the original quasiparameterization that is incomparable to or greater than any in the cofinal chain.
  Because the quasiparameterization is directed, we only need to worry about the latter case, where a slice is greater than any in the cofinal chain.
  However, if it were greater than any, it would have been included the moment it was encountered in the iteration through all parameter values.
\end{proof}

Any cofinal subset of a parameterization is itself a parameterization after suitably indexing its elements by binary strings.
However, the corresponding subset of the original index set need not be decidable.
This follows from Theorem~\ref{thm:computable_hierarchy}.
If the subset of parameter values induced by every subparameterization were decidable, we could lower the number of alternations to $\asNat(x)$.
For this, set the $\asNat(k)$th element of the subparameterization to one of the original after which no decisions about the first $\asNat(k)$ instances changes.
We can even force the subset of the index set corresponding to a subparameterization to be particularly undecidable.
The subset can be made so that it nor its complement contains an infinite semidecidable set.
In other words, there is a parameterization such that the subset of its index set corresponding to a particular subparameterization is bi-immune \parencite{rogers1967theory,odifreddi1992classical}.\indexkey{bi-immune}

For this, it is helpful to have a linear ordering on the subsets of $\binary^+$.
Most conveniently, this order comes about when we map a set $A \subseteq \binary^+$ to the real number
\begin{equation*}
  \asReal(A) \deq \sum_{x \in A} 2^{-\asNat(x)}.
\end{equation*}
That this mapping is not injective is a technical detail that is of no consequence to our analysis.
When a set $A$ is the limit of a parameterized procedure, the real number $\asReal(A)$ is said to be \emph{computably approximable} \parencite{ambos-spies2000weakly}.
Historically, computably approximable real numbers have also been called \emph{limit computable} real numbers \parencite{gold1965limiting}.
Justifying this nomenclature, a parameterized procedure $\phi$ that converges to a set $A$, can be thought of as approximating $\asReal(A)$.
Considering, for a parameter value $k$, the set
\begin{equation*}
  A_{\phi, k} \deq \{x \st \phi(x, k) = \bits{1}\},
\end{equation*}
we find that the set of real numbers $\{\asReal(A_{\phi, k}) \st k \in \binary^+\}$ has a single limit point, namely $\asReal(A)$.
Moreover, if the parameterized procedure is $f$"~alternating for some computable function $f$, the real number is said to be \emph{$\omega$"~computably enumerable}~\parencite{ambos-spies2000weakly}.
Thus, the class of computably approximable real numbers is $\Delta^0_2$, while $\Delta^{-1}_\omega$ is the class of $\omega$"~computably enumerable real numbers.

A parameterized procedure $\phi$ is called \emph{normed} \parencite{ambos-spies2000weakly} if it satisfies, for all $x$ and $k$,
\begin{equation*}
  \phi(x, k) = \bits{1} \:\implies\: \asNat(x) \le \length{k}.
\end{equation*}
This definition is practically motivated, as it holds no information about the limit of the parameterized procedure.
\begin{lemma}
\label{lem:slow_convergence_normed}%
  Let $\phi$ be an $f$"~alternating parameterized procedure that converges to a set $A$.
  There exists a normed $(1 + f)$"~alternating parameterized procedure that converges to $A$ and, on every input $(x, k)$, terminates within $\bigO(\length{k})$ steps.
  Here, the hidden constant does not depend on $\phi$.
\end{lemma}
\begin{proof}
  This lemma is a minor modification of Theorem~\ref{thm:slow_convergence}.
  The parameterized procedure constructed in the proof of that theorem can be made into a normed parameterized procedure.
  To that end, on input $(x, k)$, the procedure first checks whether $\asNat(x)$ is at most $\length{k}$.
  If not, it outputs \bits{?} and does not proceed any further.
  The extra computation can be performed in a number of steps bounded linearly in $\length{k}$.
\end{proof}

Given a normed parameterized procedure $\phi$ and two parameter values $k, k'$, it is decidable whether $\asReal(A_{\phi, k}) \le \asReal(A_{\phi, k'})$ holds.
In the case where the running time of $\phi$ is bounded linearly in the length of the parameter value, such a decision can be made within a stringent time bound.
Specifically, a decision can be made in a number of steps that is polynomial in $\length{k} + \length{k'}$, where the polynomial does not depend on $\phi$.

\begin{lemma}
\label{lem:subparameterization_bi-immune}%
  There is a parameterization with a subparameterization of which the corresponding subset of parameter values is bi-immune.
\end{lemma}
\begin{proof}
  Let $A$ be a set such that $\asReal(A)$ is computably approximable, but not $\omega$"~computably enumerable.
  Because the inclusion of $\Delta^{-1}_\omega$ in $\Delta^0_2$ is strict, such a set $A$ exists.
  Furthermore, let $\phi$ be a normed parameterized procedure converging to $A$ with a running time bounded linearly in the length of the parameter value.
  Such a procedure exists by Lemma~\ref{lem:slow_convergence_normed}.
  Lastly, define a set of parameter values
  \begin{equation*}
    C \deq \{k \st \asReal(A_{\phi,k}) < \asReal(A)\},
  \end{equation*}
  reminiscent of a Dedekind cut.
  Since the set of real numbers $\{\asReal(A_{\phi, k}) \st k \in \binary^+\}$ has a single limit point, $C$ defines a subparameterization of the parameterization
  \begin{equation*}
    (\{x \st \phi(x, k) = A(x) \reland \asNat(x) \le \length{k}\})_{k \in \binary^+}
  \end{equation*}
  on which $\phi$ converges to $A$.
  We claim that both $C$ and its complement do not contain any infinite semidecidable subset, and thus that $C$ is bi-immune.

  Because $\phi$ is normed, the order on the real numbers associated with the parameter values is decidable.
  Moreover, the set of real numbers associated with $C$ contains no greatest element.
  This allows an argument similar to the one used for Lemma~\ref{lem:cofinal_chain}.
  If $C$ contains an infinite semidecidable subset, there is an infinite decidable set $\{k_1, k_2, k_3, \ldots\} \subseteq C$ satisfying
  \begin{equation*}
    \forall i < j\colon \asReal(A_{\phi, k_i}) < \asReal(A_{\phi, k_j}).
  \end{equation*}
  Now, consider a parameterized procedure that maps $(x, \asStr(i))$ to $\phi(x, k_i)$.
  The approximation of $\asReal(A)$ corresponding to this parameterized procedure is a monotonically increasing one.
  Accordingly, for the computable function $f$ that maps $x$ to $2^{\asNat(x)}$, this parameterized procedure is $f$"~alternating.
  However, $A$ was defined so that it was not the limit of an $f$"~alternating parameterized procedure for any computable function $f$.
  Hence $C$ does not contain an infinite semidecidable subset.
  A similar argument, inverting the order, holds for the complement of $C$, completing the proof that $C$ is bi-immune.
\end{proof}

The details of the above proof are akin to those used by \textcite{jockusch1968semirecursive} in his study of sets with `selector' functions.
Of interest to us is a class of sets that arises when such functions are equipped with a polynomial bound on their running time  \parencite{selman1979p-selective}.
\begin{definition}
  A set $A$ is \defkeyat{p-selective@\pdash{}selective}{\pdash{}selective} if there is a function $f\colon \binary^+ \times \binary^+ \to \binary^+$ that is computable in polynomial time and satisfies, for all strings $x, y$,
  \begin{itemize}
  \item $f(x, y) \in \{x, y\}$, and
  \item $x \in A \lor y \in A \implies f(x, y) \in A$.
  \end{itemize}
\end{definition}

Exploiting the decidability of the order on the real numbers associated with parameter values further, we find a strengthening of Lemma~\ref{lem:subparameterization_bi-immune}.
\begin{theorem}
\label{thm:subparameterization_p-selective}%
  There is a parameterization with a subparameterization of which the corresponding subset of parameter values is bi-immune, yet \pdash{}selective.
\end{theorem}
\begin{proof}
  The bi-immune set $C$ constructed in the proof of Lemma~\ref{lem:subparameterization_bi-immune} is also \pdash{}selective.
  Given parameter values $k_1, k_2$, we can find which is associated with the smallest real number in a number of steps that is bounded polynomially in $\length{k_1} + \length{k_2}$.
  If any of the parameter values is in $C$, the one associated with the smallest real number is.
\end{proof}

The existence of a \pdash{}selective bi-immune set was claimed before \parencite{goldsmith1993note}.
However, the proof that was provided was a convoluted finite injury priority argument \parencite[see also][Section~2.11]{downey2010algorithmic}.
In our parameterized framework, such arguments can be made more transparent, as is demonstrated by the proofs of Lemma~\ref{lem:subparameterization_bi-immune} and Theorem~\ref{thm:subparameterization_p-selective}.
The number of alternations of a parameterized procedure is related to the number of injuries in a finite injury construction.
In that sense, our proof does not differ from the original proof.
However, we feel that by their framing, our proofs contribute more to an intuitive understanding of why the statements are true.
