{\begin{otherlanguage}{westernfrisian}
\chapter*{Gearfetting \\*
  \hspace*{\fill} \normalsize \mdseries \slshape Oerset troch Janneke Spoelstra
  \vspace{-\baselineskip}
}
\addcontentsline{toc}{chapter}{Gearfetting}
\markboth{Gearfetting}{Gearfetting}

Kompleksiteit kin in protte foarmen hawwe, dochs is der net ien wiskundige definysje fan kompleksiteit d{\^e}r't se allegearre oan foldogge.
Yn dit proefskrift yntrodusearje wy in wiskundich ramt foar de analyze fan ferskillende foarmen fan kompleksiteit.

Us ramt is in fuortsetting fan de parametrisearre oanpak fan komputasjonele kompleksiteit sa't Downey en Fellows dat earder dien hawwe.
Sa giet it yn dit proefskrift ek net allinnich om de analyze fan kompleksiteit, mar ek om de teory fan parametrisearre komputasjonele kompleksiteit.
Der binne twa typen resultaten yn dit proefskrift: resultaten oangeande it tapassen fan in unifoarme analyze fan kompleksiteit mei gebr{\^u}k fan {\'u}s ramt, en resultaten oangeande de klasse fan goed te behappen samlingen foar in f{\^e}ste parameter, \cl{FPT}.

Twa konkrete domeinen d{\^e}r't wy de kompleksiteit yn analysearje mei gebr{\^u}k fan {\'u}s ramt binne statistyske ynferinsje en algoritme\-{\^u}ntwerp.
Foar statistyske ynferinsje jout {\'u}s ramt rjochtlinen om it lykwicht te bewarjen tusken underfitting en overfitting fan statistyske modellen.
Us ramt biedt yndikatoaren foar algoritme\-{\^u}ntwerp, dy't br{\^u}kt wurde kinne by it besluten fan hoefolle berekkening oft der wannear en w{\^e}r dien wurde moat, foar minimaal totaal gebr{\^u}k fan helpboarnen.

Oangeande de aard fan \cl{FPT} hawwe wy trije resultaten.
Mei gebr{\^u}k fan in rangoarder fan parametrisaasjes {\^o}flaat fan \cl{FPT} litte wy sjen dat der faak gjin b{\^e}ste parametrisaasje is {\^u}nder dy dy't in samling yn \cl{FPT} pleatse.
Twad litte wy sjen dat der in strikte hi{\"e}rargy is {\^u}nder \cl{FPT} dy't basearre is op polynomiale kernelisaasjes.
As l{\^e}ste fine wy bewiis foar in alternative karakterisearring fan \cl{FPT} as de kosjintgroep $\Delta^0_1 / \cl{P}$ gebr{\^u}k meitsjend fan it symmetrysk ferskil.
Hjir is $\Delta^0_1$ de klasse fan besl{\'u}tbere samlingen, it earste trochsneednivo fan de rekkenkundige hi{\"e}rargy.

\end{otherlanguage}}
